%!TEX ROOT=formularioMatematica.tex

\section{Studio di funzione}
Lo studio di funzione è una tecnica per ricavare il grafico di una funzione che viene fornita. Questo
si basa su edi teoremi fondamentali e su tutte le conoscenze pregresse.

\subsection{Teoremi fondamentali del calcolo differenziale}
I teoremi fondamentali sono 3. Ciascuno di essi prende una parte più generale del precedente.
\subsubsection{Teorema di Rolle}
\begin{rolle}\hypertarget{teor:rolle}{}
  Sia $f$ una funzione definita e continua in $[a,b]$ e derivabile in $]a,b[$ e inoltre si abbia
  $f(a) = f(b)$. Allora
  \begin{equation*}
    \exists\,x_0\in{]a,b[}\suchthat f'(x_0)=0
  \end{equation*}
\end{rolle}

\subsubsection{Teorema di Lagrange}
\begin{lagrangeDef}\hypertarget{teor:lagrange}{}
  Sia $f$ una funzione definita e continua in $[a,b]$ e derivabile in $]a,b[$. Allora
  \begin{equation*}
    \exists\,x_0\in]a,b[\suchthat f'(x_0) = \frac{f(b)-f(a)}{b-a}
  \end{equation*}
\end{lagrangeDef}
Il teorema di Lagrange contiene al suo interno anche due lemmi (o corollari)

\begin{lagrangeLemma1}\hypertarget{teor:lagrange:1}{}
  Sia $f$ una funzione definita e continua in $I=[a,b]$ e derivabile in $\dot{I}=]a,b[$ e tale che
  \begin{equation*}
    \forall x\in\dot{I} \Rightarrow f'(x)>0
  \end{equation*}
  allora $f$ è \textbf{crescente} in $I$ e tale che
  \begin{equation*}
    \forall x\in\dot{I} \Rightarrow f'(x)<0
  \end{equation*}
  allora $f$ è \textbf{decrescente} in $I$.
\end{lagrangeLemma1}

\begin{lagrangeLemma2}\hypertarget{teor:lagrange:2}{}
  Sia $f$ una funzione definita e continua in $I=[a,b]$ con derivata nulla in $\dot{I}=]a,b[$, 
  allora
  \begin{equation*}
    f(x) = k
  \end{equation*}
\end{lagrangeLemma2}

\subsection{Teoremi sulle derivate seconde}
Il teorema di Lagrange e il suo primo lemma associa la monotonia al segno della derivata prima.
Ci sono però dei teoremi che associano la derivata seconda alla concavità.
\begin{derivataSeconda1}\hypertarget{teor:derSec:1}
  Sia $x_0\in]a,b[$ e $f$ derivabile nello stesso intorno $n$-volte. Se in $x_0$ si ha
  \begin{equation*}
    f'(x_0)=f''(x_0)=\dotsb=f^{(n-1)}(x_0)=0\land f^{(n)}(x_0)\neq0
  \end{equation*}
  si hanno i seguenti casi
  \begin{description}
    \item[$n$ pari] se
      \begin{description}
        \item[$f^{(n)}(x_0)>0$] $x_0$ è un minimo relativo
        \item[$f^{(n)}(x_0)<0$] $x_0$ è un massimo relativo
      \end{description}
    \item[$n$ dispari] in $x_0$ è presente un flesso a tangente orizzontale
  \end{description}
\end{derivataSeconda1}
\begin{derivataSeconda2}\hypertarget{teor:derSec:2}
  Sia $f$ una funzione tale che $\exists\,f''(x)$
  \begin{description}
    \item[Se $f''(x_0)>0$] ha concavità verso l'alto
    \item[Se $f''(x_0)<0$] ha concavità verso il basso
    \item[Se $f''(x_0)=0$ e $f'''(x_0)\neq0$] in $x_0$ ha un flesso
  \end{description}
\end{derivataSeconda2}

\subsection{Esempio}
Avendo ora tutte le cose necessarie per farlo, definiamo i passaggi per studiare una funzione
\begin{enumerate}
  \item Dominio
  \item Intersezione con gli assi
  \item Simmetria
  \item Periodicità
  \item Segno
  \item Asintoti
  \item Continuità e derivabilità
  \item Massimi e minimi
  \item Concavità e flessi
\end{enumerate}
Per chiarire il tutto, prendiamo ad esempio la seguente funzione
\begin{equation*}
  f(x) = \frac{x+2}{x^2-x}
\end{equation*}
Per prima cosa quindi troviamo il dominio
\begin{equation*}
  x^2-x\neq0 \rightarrow x(x-1)\neq 0
\end{equation*}
quindi
\begin{equation*}
  \mathcal{D} = \mathbb{R} \setminus \{0,1\}
\end{equation*}
\begin{center}
  \begin{tikzpicture}
    \coordinate (O) at (0,0);
    \coordinate (A) at (1,0);
    \domainplot{O}{A}{{0,1}}
  \end{tikzpicture}
\end{center}
