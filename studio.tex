%!TEX ROOT=formularioMatematica.tex

\section{Studio di funzione}
Lo studio di funzione è una tecnica per ricavare il grafico di una funzione che viene fornita. Questo
si basa su edi teoremi fondamentali e su tutte le conoscenze pregresse.

\subsection{Teoremi fondamentali del calcolo differenziale}
I teoremi fondamentali sono 3. Ciascuno di essi prende una parte più generale del precedente.
\subsubsection{Teorema di Rolle}
\begin{rolle}\hypertarget{teor:rolle}{}
  Sia $f$ una funzione definita e continua in $[a,b]$ e derivabile in $]a,b[$ e inoltre si abbia
  $f(a) = f(b)$. Allora
  \begin{equation*}
    \exists\,x_0\in{]a,b[}\suchthat f'(x_0)=0
  \end{equation*}
\end{rolle}

\subsubsection{Teorema di Lagrange}
\begin{lagrangeDef}\hypertarget{teor:lagrange}{}
  Sia $f$ una funzione definita e continua in $[a,b]$ e derivabile in $]a,b[$. Allora
  \begin{equation*}
    \exists\,x_0\in]a,b[\suchthat f'(x_0) = \frac{f(b)-f(a)}{b-a}
  \end{equation*}
\end{lagrangeDef}
Il teorema di Lagrange contiene al suo interno anche due lemmi (o corollari)

\begin{lagrangeLemma1}\hypertarget{teor:lagrange:1}{}
  Sia $f$ una funzione definita e continua in $I=[a,b]$ e derivabile in $\dot{I}=]a,b[$ e tale che
  \begin{equation*}
    \forall x\in\dot{I} \Rightarrow f'(x)>0
  \end{equation*}
  allora $f$ è \textbf{crescente} in $I$ e tale che
  \begin{equation*}
    \forall x\in\dot{I} \Rightarrow f'(x)<0
  \end{equation*}
  allora $f$ è \textbf{decrescente} in $I$.
\end{lagrangeLemma1}

\begin{lagrangeLemma2}\hypertarget{teor:lagrange:2}{}
  Sia $f$ una funzione definita e continua in $I=[a,b]$ con derivata nulla in $\dot{I}=]a,b[$, 
  allora
  \begin{equation*}
    f(x) = k
  \end{equation*}
\end{lagrangeLemma2}

\subsubsection{Teorema di Cauchy}
\begin{cauchy}
  Siano $f$ e $g$ due funzioni continue e definite in $[a,b]$ e derivabili in $]a,b[$ con
  $g'(x)\neq0\,\forall x\in]a,b[$; allora
  \begin{equation*}
    \exists\,c\in]a,b[\suchthat \frac{f'(c)}{g'(c)}=\frac{f(b)-f(a)}{b-a}
  \end{equation*}
\end{cauchy}
Da questo teorema, si ricava uno dei principali teoremi per il calcolo di limiti.
\begin{hopital}
  Siano $f$ e $g$ continue in $[a,b]$, derivabile in $]a,b[\setminus\{x_0\}$ con 
  $g'(x)\neq0\forall x\in]a,b[$ e $f(x_0)=0 \land g(x_0)=0$ o $f(x_0)=\infty\land g(x_0)=\infty$.
  Se esiste finito
  \begin{equation*}
    \lim\limits_{x\to x_0} \frac{f'(x)}{g'(x)}
  \end{equation*}
  allora esiste
  \begin{equation*}
    \lim\limits_{x\to x_0} \frac{f(x)}{g(x)}
  \end{equation*}
  e si ha
  \begin{equation*}
    \lim\limits_{x\to x_0} \frac{f'(x)}{g'(x)} = \lim\limits_{x\to x_0} \frac{f(x)}{g(x)}
  \end{equation*}
\end{hopital}
Con questo teorema si possono calcolare tutte le forme indeterminate, con opportune modifiche. 
Infatti se si ha uno $^0\!/_0$ o un $^\infty\!/_\infty$ si risolve direttamente. Se si ha un
$0\infty$ si può ricondurre facilmente. Se si ha un $\pm\infty\pm\infty$
si fa il limite di
\begin{equation*}
  f \left[ 1-\frac{g}{f} \right]
\end{equation*}
o se è uno $^0\!/_0$ si faccia il denominatore comune. Infine tutte le $0^0$, $1^\infty$ e
$\infty^\infty$ si riconducono alla $0\infty$.

\subsection{Teoremi sulle derivate seconde}
Il teorema di Lagrange e il suo primo lemma associa la monotonia al segno della derivata prima.
Ci sono però dei teoremi che associano la derivata seconda alla concavità.
\begin{derivataSeconda1}\hypertarget{teor:derSec:1}
  Sia $x_0\in]a,b[$ e $f$ derivabile nello stesso intorno $n$-volte. Se in $x_0$ si ha
  \begin{equation*}
    f'(x_0)=f''(x_0)=\dotsb=f^{(n-1)}(x_0)=0\land f^{(n)}(x_0)\neq0
  \end{equation*}
  si hanno i seguenti casi
  \begin{description}
    \item[$n$ pari] se
      \begin{description}
        \item[$f^{(n)}(x_0)>0$] $x_0$ è un minimo relativo
        \item[$f^{(n)}(x_0)<0$] $x_0$ è un massimo relativo
      \end{description}
    \item[$n$ dispari] in $x_0$ è presente un flesso a tangente orizzontale
  \end{description}
\end{derivataSeconda1}
\begin{derivataSeconda2}\hypertarget{teor:derSec:2}
  Sia $f$ una funzione tale che $\exists\,f''(x)$
  \begin{description}
    \item[Se $f''(x_0)>0$] ha concavità verso l'alto
    \item[Se $f''(x_0)<0$] ha concavità verso il basso
    \item[Se $f''(x_0)=0$ e $f'''(x_0)\neq0$] in $x_0$ ha un flesso
  \end{description}
\end{derivataSeconda2}

\subsection{Esempio}
Avendo ora tutte le cose necessarie per farlo, definiamo i passaggi per studiare una funzione
\begin{enumerate}
  \item Dominio
  \item Intersezione con gli assi
  \item Simmetria
  \item Periodicità
  \item Segno
  \item Asintoti
  \item Continuità e derivabilità
  \item Massimi e minimi
  \item Concavità e flessi
\end{enumerate}
Per chiarire il tutto, prendiamo ad esempio la seguente funzione
\begin{equation*}
  f(x) = \frac{x^2-5x+4}{x-5}
\end{equation*}
Per prima cosa quindi troviamo il dominio
\begin{equation*}
  x^2-x\neq0 \rightarrow x(x-1)\neq 0
\end{equation*}
quindi
\begin{equation*}
  \mathcal{D} = \mathbb{R} \setminus \{0,1\}
\end{equation*}
\begin{center}
  \begin{tikzpicture}
    \coordinate (O) at (4,0);
    \coordinate (A) at (6,0);
    \domainplot{O}{A}{{5}}
    %\drawSign{(\x^2-5*\x+4)/(\x-5+0.001)}{-2}{7}{0.25}{5}
  \end{tikzpicture}
\end{center}
Trovato il dominio, troviamo le intersezioni con gli assi
\begin{equation*}
  \begin{cases}
    y=0\\
    y = \frac{x^2-5x+4}{x-5}
  \end{cases}
  \begin{cases}
    y = 0\\
    x-5x+4=0
  \end{cases}
  \begin{cases}
    x_{1/2} =
    \begin{cases}
      x_1 = 4\\ x_2=1
    \end{cases}
  \end{cases}
\end{equation*}
Quindi i due punti di intersezione con $x$ sono
\begin{equation*}
  A(4,0)\quad B(1,0)
\end{equation*}
Ora con $y$
\begin{equation*}
  \begin{cases}
    x=0\\
    y=-\frac{4}{5}
  \end{cases}
\end{equation*}
e quindi il punto è immediato
\begin{equation*}
  C \left( -\frac{4}{5},0 \right)
\end{equation*}
Prima di andare a disegnare queste informazioni, troviamone altre due che miglioreranno enormemente
il disegno. La prima sono gli asintoti.\\
Troviamo gli eventuali asintoti verticali. Dato che $5\notin\mathcal{D}$,
\begin{equation*}
  \lim\limits_{x\to5^-} \frac{x^2-5x+4}{x-5} = \frac{4}{0^-} = -\infty
\end{equation*}
e
\begin{equation*}
  \lim\limits_{x\to5^+} \frac{x^2-5x+4}{x-5} = \frac{4}{0^+}= +\infty
\end{equation*}
Sappiamo quindi che $x=5$ è \textbf{asintoto verticale}.\\
Andiamo alla ricerca di un asintoto orizzontale (anche se vedendo la funzione possiamo subito
vedere che non è presente in quanto il grado del numeratore è maggiore di quello del denominatore)
\begin{equation*}
  \lim\limits_{x\to\infty} \frac{x^2-5x+4}{x-5} = 
  \lim\limits_{x\to\infty} 
  \frac{x^2\left(-\frac{5}{x}+\frac{4}{x^2}\right)}{x\left(1-\frac{5}{x}\right)}=
  \infty
\end{equation*}
Quindi non ci sono asintoti orizzontali e dobbiamo andare a cercarne di obliqui quindi
\begin{equation*}
  q = \lim\limits_{x\to\infty} \frac{x^2-5x+4}{x-5}\cdot \frac{1}{x} =
  \lim\limits_{x\to\infty} \frac{x^2-5x+4}{x^2-5x} = 1
\end{equation*}
\begin{align*}
  m &= \lim\limits_{x\to\infty} \left[\frac{x^2-5x+4}{x-5}-x\right]=
  \lim\limits_{x\to\infty} \frac{x^2-5x+4-5x^2+5x}{x-5}=\\
  &\lim\limits_{x\to\infty} \frac{4}{x-5}=0
\end{align*}
Questo significa che $y=x$ è \textbf{asintoto obliquo}.\\
Prima di disegnare, verifichiamo che i conti siano corretti andando a calcolare il segno della 
funzione
\begin{center}
  \begin{tikzpicture}
    \drawSign{(\x^2-5*\x+4)/(\x-5+0.001)}{-1}{7}{0.25}{5} 
  \end{tikzpicture}
\end{center}
  Vediamo che effettivamente le informazioni degli asintoti e quella del segno combaciano. Prima
  di $5$ la funzione tende a $-\infty$ quindi è negativa, poi diventa positiva. Vediamo anche che
  $1$ e $4$ sono punti d'intersezione e quindi che il segno cambia. Disegnamo ciò che sappiamo
  \begin{center}
    \begin{tikzpicture}
      \begin{axis}[xmin=-3,ymin=-5,xmax=10,ymax=20]
        \coordinate (O) at (0,0);
        \node[fill=white,circle,inner sep=0pt] (O-label) at ($(O)+(-135:10pt)$) {$O$};
        \addplot[green,domain=-5:18] {x};
        \addplot[red,thick,samples=51,unbounded coords=jump,domain=-5:-2.5] {(x^2-5*x+4)/(x-5)};
        \addplot[red,thick,samples=51,unbounded coords=jump,domain=5.1:5.4] {(x^2-5*x+4)/(x-5)};
        \addplot[red,thick,samples=51,unbounded coords=jump,domain=15:17.7] {(x^2-5*x+4)/(x-5)};
        \addplot[red,thick,samples=51,unbounded coords=jump,domain=4.5:4.9] {(x^2-5*x+4)/(x-5)};
        \addplot+[blue,mark=none] coordinates {(5,-5) (5,20)};
        \LabelPoint[mark=*,mark size=1.5][above]{4}{0}{$A$}
        \LabelPoint[mark=*,mark size=1.5][above]{1}{0}{$B$}
        \LabelPoint[mark=*,mark size=1.5][below right]{0}{-4/5}{$C$}
      \end{axis}
    \end{tikzpicture}
  \end{center}
  Questo rapido grafico contiene le informazioni che abbiamo già trovato, i due asintoti e il segno.
  Sono anche riporati i punti di intersezione precedentemente trovati.\\
  A questo punto vediamo che la funzione è continua in quanto è polinomiale. Andiamo a vedere se
  è derivabile e qual è la sua derivata prima.
  \begin{align*}
    f'(x) &= \frac{\Dif(x^2-5x+4)(x-5)-(x^2-5x+4)\Dif(x-5)}{(x-5)^2} =\\
          &= \frac{(2x-5)(x-5)-(x^2-5x+4)}{(x-5)^2} = \frac{x^2-10x+21}{(x-5)^2}
  \end{align*}
  Trovata la derivata, studiando il segno si vede che
  \begin{center}
    \begin{tikzpicture}
      \drawSign[scale=0.75]{(\x^2-10*\x+21)/(\x+5.001)^2}{0}{10}{0.25}{}
      \end{tikzpicture}
    \end{center}
    Questo significa che in $x=3$ e in $x=7$ cambia segno. La funzione da crescente diventa decrescente
    e viceversa. Questo significa che se la derivata calcolata in quel punto è pari a $0$, sono
    un punto di minimo e uno di massimo.
    \begin{align*}
      f'(3) &= \frac{9-30+21}{(3-5)^2} = 0\\
      f'(7) &= \frac{49-70+21}{(7-5)^2} = 0
    \end{align*}
    E quindi in $3$ e $7$ si hanno rispettivamente un punto di massimo e uno di minimo. Calcoliamone il
    valore della $y$ e completiamo lo studio
    \begin{equation*}
      m \left( 3,1 \right)\quad M \left( 7,9 \right)
    \end{equation*}
    Quindi il disegno ora completo è
    \begin{center}
      \begin{tikzpicture}
        \begin{axis}[xmin=-3,ymin=-5,xmax=10,ymax=20]
          \coordinate (O) at (0,0);
          \node[fill=white,circle,inner sep=0pt] (O-label) at ($(O)+(-135:10pt)$) {$O$};
          \addplot[green,domain=-5:18] {x};
          \addplot[red,thick,samples=91,smooth,unbounded coords=jump,domain=-5:4.90] 
            {(x^2-5*x+4)/(x-5)};
          \addplot[red,thick,samples=91,smooth,unbounded coords=jump,domain=5.10:18] 
            {(x^2-5*x+4)/(x-5)};
          \addplot+[blue,mark=none] coordinates {(5,-5) (5,20)};
          \LabelPoint[mark=*,mark size=1.5][above]{4}{0}{$A$}
          \LabelPoint[mark=*,mark size=1.5][above]{1}{0}{$B$}
          \LabelPoint[mark=*,mark size=1.5][below right]{0}{-4/5}{$C$}
          \LabelPoint[mark=*,mark size=1.5][above]{3}{1}{$m$}
          \LabelPoint[mark=*,mark size=1.5][below left]{7}{9}{$M$}
        \end{axis}
      \end{tikzpicture}
    \end{center}
