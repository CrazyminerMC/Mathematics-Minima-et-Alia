%!TEX ROOT=formularioMatematica.tex

\section{Equazioni differenziali}
Se nell'algebra tradizionale si lavora con variabiili come $x$ che rappresentanoun numero, non è
l'unica possibilità. Se si mettono in relazione una variabile $x$, una funzione $f(x)$ e sue
derivate successive, si ottiene un'equazione differenziale. Essa è spesso espressa nella forma
\begin{equation*}
  F\left(x,y,y',y'',\ldots,y^{(n)}\right)=0
\end{equation*}
Si definisce \textbf{ordine} di un'equazione differenziale il massimo ordine di derivazione. Si
definisce \textbf{grado} il grado della funzione di ordine massimo.\\
Ci sono due tipi di soluzioni: \textbf{generale} ovvero una funzione che soddisfa la relazione con la
presenza di una costante (essa sarà nella forma $y=\varphi(x,c_1,c_2,\ldots,c_n)$) e 
\textbf{particolare} ovvero una funzione che soddisfa la relazione senza costante. Per questo secondo
tipo, sono necessarie ulteriori condizioni.\\
Se si riesce a scrivere un'equazione differenziale nella forma
\begin{equation*}
  y^{(n)}=G(x,y,y',\ldots,y^{(n-1)})
\end{equation*}
si dice che è scritta in \textbf{forma normale}.

\subsection{Equazioni differenziali a variabili separabili}
Se nell'equazione
\begin{equation*}
  y^{(n)}=G(x,y) 
\end{equation*}
è possibile scrivere $G(x,y)$ come $M(x)\cdot N(y)$, allora si definisce un'equazione differenziale a
variabili separabili. Per risolverla quindi si ha che
\begin{align*}
  y' &= M(x)\cdot N(y)\\
  \int \frac{y'}{N(y)}\,\dif x &= \int M(x)\,\dif x
\end{align*}

\subsection{Problema di Cauchy}
Dato il sistema
\begin{equation*}
  \begin{cases}
    y' = G(x,y)\\
    y(x_0) = y_0
  \end{cases}
\end{equation*}
Esiste ed è unica la soluzione. Ovviamente per un maggiore ordine, sono necessarie maggiori 
condizioni se si vogliono eliminare tutte le costanti.

\subsection{Equazioni differenziali lineari di primo ordine}
Sia data
\begin{equation*}
  y'+a(x)y=b(x)
\end{equation*}
Se $b(x)=0$ si definisce lineare \textbf{omogenea}. Per risolvere questo tipo di equazioni si deve
moltiplicare tutto per un fattore. Esso è
\begin{equation*}
  e^{\int a(x)\,\dif x}
\end{equation*}
Quindi l'equazione diventa
\begin{equation*}
  \underbrace{y'e^{\int a(x)\,\dif x}+a(x)e^{\int a(x)\,dif x}}_
  {\frac{\dif}{\dif x}\left[ ye^{\int a(x)\,\dif x} \right]}=b(x)e^{\int a(x)\,\dif x}
\end{equation*}
E quindi possiamo scrivere semplicemente
\begin{equation*}
  \int \frac{\dif}{\dif x}\left[ ye^{\int a(x)\,\dif x} \right]\,\dif x = 
  \int b(x)e^{\int a(x)\,\dif x}\,\dif x
\end{equation*}
e infine, isolando
\begin{equation*}
  y = e^{-\int a(x)\,\dif x}\int b(x)e^{\int a(x)\,\dif x}\,\dif x
\end{equation*}

\subsection{Equazioni differenziali lineari di secondo ordine}
Sia data
\begin{equation*}
  ay''+by'+cy=g(x) 
\end{equation*}
Se $g(x)=0$ si definisce lineare \textbf{omogenea}. Dato che $a$, $b$, $c$ sono coefficienti 
costanti, si può dividere tutto per $a$ e ottenere
\begin{equation*}
  y''+\frac{b}{a}y'+\frac{c}{a}y=0
\end{equation*}
che si riscrive in
\begin{equation*}
  y''+by'+c=0
\end{equation*}
Le soluziooni sono del tipo $y=e^{\lambda x}$ e quindi si ha che
\begin{equation*}
  y = e^{\lambda x}\quad y'=\lambda e^{\lambda x}\quad y''=\lambda^2 e^{\lambda x}
\end{equation*}
E quindi
\begin{equation*}
  \lambda^2 e^{\lambda x}+b\lambda e^{\lambda x}+ce^{\lambda x}=0
\end{equation*}
e raccogliendo
\begin{equation*}
  e^{\lambda x}(\lambda^2+b\lambda+c)=0
\end{equation*}
Questo accade solo se $\lambda^2+b\lambda+c=0$ visto che 
$\forall \lambda\in\mathbb{R}, e^{\lambda x}>\neq 0$. $\lambda^2+b\lambda+c=0$ si definisce
\textbf{equazione caratteristica}. Per risolvere queste equazioni differenziali, si distingue in base
al delta dell'equazione caratteristica.
\subsubsection{$\Delta > 0$}
Se $\Delta>0$ allora si avranno due soluzioni $e^{\lambda_1 x}\land e^{\lambda_2 x}$. La soluzione
generale quindi sarà $y=Ae^{\lambda_1 x}+Be^{\lambda_2 x}$ per qualche $A$ e $B$.
\subsubsection{$\Delta = 0$}
Se $\Delta=0$ allora si avranno due soluzioni coincidenti. Quindi la soluzione generale sarà nella 
forma $y=Ae^{\lambda x}+Bxe^{\lambda x}$ per qualche $A$ e $B$.
\subsubsection{$\Delta<0$}
Se $\Delta<0$ si avranno due soluzioni distinte, complesse coniugate 
$\lambda_{1/2}=\alpha+ i\beta$. Utilizzando la formula di Eulero per i numeri complessi, la 
soluzione generale diventa $y=Ae^{\alpha+i\beta}+Be^{\alpha+i\beta}$ che diventa, prendendo le parti
reali $y=Ae^{\alpha x}\cos\beta x +Be^{\alpha x}\sin\beta x$ per qualche $A$ e $B$.\\ [\baselineskip]
Se l'equazione non è omogenea, si avrà la soluzione generale come
\begin{equation*}
  y = y_{\text{omogenea}}+y_{\text{particolare}}
\end{equation*}
dove $y_\text{particolare}$ si ricava tramite tabelle.
