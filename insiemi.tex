%!TEX ROOT=formularioMatematica.tex

\section{Insiemi numerici}\label{sec:insiemi}
Se durante la nostra carriera scolastica abbiamo sempre lavorato con $\mathbb{R}$ o al massimo in 
$\mathbb{C}$, nulla vieta che noi creiamo nuovi insiemi numerici e li studiamo per trovarne alcune
caratteristiche.\\
Ogni insieme numerico possiede delle caratteristiche che noi possiamo studiare
\begin{enumerate}
	\item � limitato/illimitato
	\item Possiede un $\max$ e un $\min$
	\item Possiede \emph{maggioranti} o \emph{minoranti}
	\item Possiede un $\sup$ o un $\inf$
\end{enumerate}
Per le seguenti definizioni ed esempi, prenderemo in considerazione
\begin{equation*}
A = \left\{\frac{1}{n}\mid n\in\mathbb{N}_0\right\}
\end{equation*}

\subsection{Condizioni di limitazione}
La o le condizioni di limitazione indicani quale pu� essere un limite o i limiti di un insieme. In 
insieme pu� essere illimitato (ovvero che per qualunque numero noi scegliamo, esister� un punto
sulla retta che lo rappresenta), limitato \emph{superiormente}, \emph{inferiormente} o entrambi
contemporaneamente.\\
In generale la condizione di limitazione (superiore ed inferiore per uno stesso valore) �
\begin{equation*}
\exists\,k>0\land k\in\mathbb{R}\mid \forall x \in A, \abs{x} \leq k
\end{equation*}
Generalizzando ancora per due valori diversi
\begin{equation*}
\exists\,k_1,k_2>0\land k_1,k_2\in\mathbb{R}\mid \forall x\in A, k_1\leq x \leq k_2
\end{equation*}

\subsection{Maggioranti e minoranti}
Prendendo le condizioni di limitazione separatamente
\begin{align*}
\exists\,m\in\mathbb{R}\mid\forall x\in A x\geq m
\intertext{e}
\exists\,M\in\mathbb{R}\mid\forall x\in A x\leq M
\end{align*}
$m$ rappresenta un \textbf{minorante} di $A$ e $M$ rappresenta un \textbf{maggiorante} di $A$.

\subsection{Massimi e minimi}
Un numero si definisce \emph{massimo} se
\begin{equation*}
\exists\,L\in\mathbb{A}\mid\forall x\in A, L\geq x
\end{equation*}
quindi � il valore pi� alto che l'insieme contiene.\\
Un numero si definisce \emph{minimo} se
\begin{equation*}
\exists\,l\in\mathbb{A}\mid\forall x\in A, l\leq x
\end{equation*}
quindi � il valore pi� basso che l'insieme contiene.

\subsection{Intervalli}
Un \emph{intervallo} pu� essere aperto (illimitato) o chiuso (limitato). La notazione pi� comune � la 
seguente
\begin{align*}
\textbf{Limitato }&{[{1},{4}]}\coloneqq\{x\in\mathbb{R}\mid1\leq x \leq4\}\\
\textbf{Illimitato }&{]{-\infty},{\pi}[}\coloneq\{x\in\mathbb{R}\mid-\infty<x<\pi\}
\end{align*}
Da notare che $\pm\infty$ � sempre escluso in quanto tecnicamente non appartiene a $\mathbb{R}$.\\
Un \emph{intorno} non � altro che un intervallo che comprende un numero specifico. In simboli
\begin{equation*}
I(x_0)\coloneqq{]{x_0-\delta_1},{x_0+\delta_2}[} \qquad(\delta_1,\delta_2\in\mathbb{R})
\end{equation*}
Ovviamente si pu� definire un intorno che sia limitato con una distanza
\begin{equation*}
I_\varepsilon(x_0)\coloneqq{]{x_0-\varepsilon},{x_0+\varepsilon}[} 
\end{equation*}
Questo � completo e ovviamente possiamo anche fare intorni non completi (quindi solo da un lato). Essi 
sono di conseguenza denominati sinistri o destri.

\subsection{Punti isolati}
Un punto isolato � un punto i cui intorni non contengono alcun elemento dell'insieme.
\begin{equation*}
x_0\in A, \exists\,I(x_0)\mid\forall x\in A\setminus\{x_0\}\not\supset I(x_0)
\end{equation*}

\subsection{Punti di accumulazione}
Un punto di accumulazione � un punto in cui ogni suo intorno cade almeno un elemento distinto 
dell'insieme.
\begin{equation*}
x_0,y\in A,\forall I(x_0), y\in I(x_0)
\end{equation*}

\subsection{Estremi}
L'estremo superiore � quel valore che non viene mai superato. A seconda dei casi pu� essere il 
\emph{pi� grande elemento dell'insieme} o il \emph{pi� piccolo dei maggioranti}.
\begin{equation*}
\forall x\in A\implies x\leq\Lambda\quad\text{e}\quad \forall\varepsilon\in\mathbb{R}_0^+,
\exists\,x\in A\mid x>\Lambda-\varepsilon
\end{equation*}
L'estremo inferiore � quel valore che non non ha valori inferiori. A seconda dei casi pu� essere il
\emph{pi� piccolo elemento dell'insieme} o il \emph{pi� grande dei minoranti}.
\begin{equation*}
\forall x\in A\implies x\geq\lambda\quad\text{e}\quad \forall\varepsilon\in\mathbb{R}_0^+,
\exists\,x\in A\mid x<\lambda+\varepsilon
\end{equation*}