%!TEX ROOT=formularioMatematica.tex

\section{Successioni}\label{sec:successioni}
Una successione è una particolare funzione definita in modo
\begin{equation*}
  f:\,\mathbb{N} \rightarrow \mathbb{R} 
\end{equation*}
ovvero
\begin{equation*}
  n \mapsto a_n=f(n)
\end{equation*}
Una successione quindi è una serie di numeri interi relazionati fra di loro. Ci sono generalmente 3
modi per definire una successione:
\begin{enumerate}
  \item \textbf{Algebrica}
    \begin{equation*}
      a_n = 2n^2+1
    \end{equation*}
  \item \textbf{Ricorsiva} 
    \begin{equation*}
      \begin{cases}
        a_0 &= 1\\ a_n &= 2a_{n-1}-1
      \end{cases}
    \end{equation*}
  \item \textbf{Elencativa}
    \begin{equation*}
      a_n=\{1,3,5,7,\ldots\}
    \end{equation*}
\end{enumerate}
Per andare a studiare una successione, calcoliamo il limite. Dato che per polimorfismo $\mathbb{N}$
ha un solo punto di accumulazione che corrisponde a $+\infty$, il solo limite che possiamo calcolare
è
\begin{equation*}
  \lim\limits_{n \to \infty} a_n
\end{equation*}
che può assumere 4 valori e a seconda del valore che ottiene, si definisce la successione in modo
diverso.
\begin{equation*}
  \lim\limits_{n \to \infty} a_n=
  \begin{cases}
    l, &\text{la successione $\{a_n\}$ è convergente}\\
    +\infty, &\text{la succcessone $\{a_n\}$ è divergente positivamente}\\
    -\infty, &\text{la successione $\{a_n\}$ è divergente negativamente}\\
    \not\exists, &\text{la successione $\{a_n\}$ è indeterminata}
  \end{cases}
\end{equation*}
La definizione formale del limite quindi cambia leggermente definizione
\begin{align*}
  \lim\limits_{n \to \infty} a_n=l \Leftrightarrow \forall\varepsilon>0,\,\exists\bar{n}_\varepsilon
  \mid\forall n>\bar{n}_\varepsilon \Rightarrow \abs{a_n-l}<\varepsilon
\end{align*}

\subsection{Teorema sulle successioni}
\begin{successioniMonotone}
  Se $a_n$ è crescente e limitata superiormente, allora ammette limite che coincide con l'estremo
  superiore.
\end{successioniMonotone}

\subsection{Serie numeriche}
Una serie numerica è una somma di una successione. Una somma infinita però. Quindi
\begin{equation*}
  \sum\limits^{\infty}_{i=1} a_i = a_1+a_2+a_3+\dotsb+a_n+\dotsb
\end{equation*}
Per studiare una serie, si studia il limite a ll'infinito ma facendolo così direttamente non è
possibile, quindi si devono creare delle somme parziali. Ad esempio
\begin{align*}
  \text{Sia }a_1,a_2,\ldots,&a_n,\ldots\text{ una successione}\\
  s_1 &= a_1\\
  s_2 &= a_1+a_2\\
  s_n &= \sum\limits^{n}_{i=1} a_i
\end{align*}
Se quindi $\{s_n\}$ è una successione di somme parziali,
\begin{equation*}
  \lim\limits_{n \to \infty} s_n = S
\end{equation*}
dove $s_n$ deve essere convergente e $S$ è la somma della serie. Generalizzando quindi
\begin{equation*}
  \sum\limits^{\infty}_{i=1} = \lim\limits_{n \to \infty} s_n = S
\end{equation*}

\subsubsection{Serie di Mengoli-Cauchy}
Questa serie è forse la più celebre e può far capire come approcciarsi alle serie
\begin{equation*}
  \sum\limits^{\infty}_{i=1} \frac{1}{i(i+1)}
\end{equation*}
Per andare a risolvere questa serie bisogna riscrivere il parametro in quanto altrimenti
il limite all'infinito avrebbe una forma indeterminata del tipo $\infty\cdot\infty$. Quindi 
possiamo scrivere
\begin{equation*}
  \frac{1}{i(i+1)} = \frac{A(i+1)+Bi}{i(i+1)} = \frac{(A+B)i+A}{i(i+1)}
\end{equation*}
dove $A$ e $B$ rappresentano i coefficienti. Per la proprietà d'identità dei polinomi scriviamo
\begin{equation*}
  \begin{cases}
    A+B=0\\A=1
  \end{cases}
  \begin{cases}
    B=-1\\A=1
  \end{cases}
\end{equation*}
Quindi
\begin{equation*}
  \frac{1}{i(i+1)} = \frac{1}{i}-\frac{1}{i+1}
\end{equation*}
Di conseguenza la nostra serie diventa
\begin{equation*}
  \sum\limits^{\infty}_{i=1} \left( \frac{1}{i}-\frac{1}{i+1} \right)
\end{equation*}
Ora quindi abbiamo riscritto la serie in modo che sia di facile verifica. Andando ora ad osservare
le somme parziali
\begin{align*}
  \label{eq:}
  s_1 &= 1-\frac{1}{2}=\frac{1}{2}\\
  s_2 &= 1-\cancel{\frac{1}{2}}+\cancel{\frac{1}{2}}-\frac{1}{3}=\frac{2}{3}\\
  s_3 &= 1-\cancel{\frac{1}{2}}+\cancel{\frac{1}{2}}-\cancel{\frac{1}{3}}+\cancel{\frac{1}{3}}
  -\frac{1}{4}=\frac{3}{4}
\end{align*}
Notiamo che $1$ rimane sempre e che rimane anche il termine $-\frac{1}{i+1}$. Quindi la
somma parziale generalizzata è
\begin{equation*}
  s_i=1-\frac{1}{i+1}
\end{equation*}
Il limite dunque diventa
\begin{equation*}
  \lim\limits_{i\to\infty}  \left(1-\frac{1}{i+1} \right) = 1
\end{equation*}
Di conseguenza
\begin{equation*}
  \sum\limits^{\infty}_{i=1} \frac{1}{i(i+1)} = 1
\end{equation*}

\subsubsection{Progressioni geometriche}
Sia $a_1,a_2,\ldots,a_n,\ldots$ una progressione geometrica i cui elementi sono
\begin{align*}
  a_2&=q_a1\\
  a_3&=q^2a_1\\
  a_4&=q^3a_1\\
  a_n&=q^{n-1}a_1
\end{align*}
Dove le somme parziali sono uguali a 
\begin{equation*}
  s_n = a_1 \frac{1-q^n}{1-q}
\end{equation*}
stando alle formule trattate nelle sezioni precedenti. Quindi la serie di una progressione
geometrica è
\begin{equation*}
  \sum\limits^{\infty}_{i=1} x^{i-1}=\sum\limits^{\infty}_{i=0} x^i
\end{equation*}
e il limite della somma parziale
\begin{equation*}
  \lim\limits_{n \to \infty} a_1 \frac{1-q^n}{1-q}=
  \frac{a_1}{1-q}\lim\limits_{n \to \infty} \left( 1-q^n \right)
\end{equation*}
che può assumere 3 valori a seconda della ragione ($q$) della progressione.
\begin{equation*}
  \frac{a_1}{1-q}\lim\limits_{n \to \infty} \left( 1-q^n \right) =
  \begin{cases}
    0, &\abs{q}<1\\
    \pm\infty, &q>1\\
    \not\exists, &q\leq-1
  \end{cases}
\end{equation*}
