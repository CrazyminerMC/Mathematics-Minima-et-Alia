%!TEX ROOT=formularioMatematica.tex

\section{Geometria analitica}\label{sec:geomanal}
La geometria analitica è la geometria che si occupa di lavorare nel piano cartesiano ($xOy$).\\
Per gli esercizi si vada a pagina~\pageref{ex:geomanal}.

\subsection{Generale}
Le formule qui riportate sono generali a tutto l'ambito della geometria analitica e non si riferiscono
ad una figura particolare.\\
Di seguito viene rappresentato il tipico piano cartesiano con i suoi quattro quadranti.
\begin{center}
  \begin{tikzpicture}
    \begin{axis}[xmin=-3,ymin=-3,xmax=3,ymax=3]
      \coordinate (O) at (0,0);
      \coordinate (P) at (2,1);
      \pgfmathsetmacro{\P}{(2,1)}
      \node[fill=white,circle,inner sep=0pt] (O-label) at ($(O)+(-135:10pt)$) {$O$};
      \LabelPoint[mark=*,color=red][color=blue,below]{2}{1}{$P(x_P,y_P)$}
    \end{axis}
  \end{tikzpicture}
\end{center}
D'ora in poi, si darà per scontata la convenzione di nominare le coordinate di un punto in base al nome
del punto stesso. Ad esempio $P(x_P, y_P)$. Si noti anche che è possibile definire un punto attraverso
un vettore bidimensionale. Ovvero
\begin{equation*}
  P(x_P,y_P) = \begin{bmatrix}
    x_P\\y_P
  \end{bmatrix}
\end{equation*}

\subsubsection{Distanza tra due punti}
\begin{equation*}
  \overline{AB} = \sqrt{(x_B-x_A)^2+(y_B-y_A)^2}
\end{equation*}

\subsubsection{Punto medio}
\begin{equation*}
  M\left(\frac{x_A-x_B}{2}, \frac{y_A-y_B}{2}\right)
\end{equation*}

\subsubsection{Punto su un segmento in un rapporto $\frac{m}{n}$}
Siano $m$ e $n$ due rapporti a cui sta un punto rispetto al segmento. Ovvero il punto $P(x_P, y_P)$
divide il segmento in due parti una lunga $\frac{n}{m+n}$ e l'altra $\frac{m}{m+n}$.
\begin{equation*}
  P\left(\frac{nx_A+mx_B}{m+n},\frac{ny_A+my_B}{m+n}\right)
\end{equation*}

\subsubsection{Baricentro di un triangolo}
\begin{equation*}
  G\left(\frac{x_A+x_B+x_C}{3},\frac{y_A+y_B+y_C}{3}\right)
\end{equation*}

\subsubsection{Area di un poligono qualsiasi}
\begin{equation*}
  \mathscr{A}(P) = \frac{1}{2}\left\lvert 
    \begin{matrix}[1]
      x_1 & y_1 & 1\\
      x_2 & y_2 & 1\\
      x_3 & y_3 & 1\\
      \vdots & \vdots & 1
  \end{matrix}\right\rvert
\end{equation*}
usando la regola di Sarrus. Questa formula è anche chiamata la formula di Gauss per le aree di 
poligoni.\\
Il modo di calcolare il determinante della matrice è il seguente (andare
in diagonale dall'alto per ogni elemento della prima colonna moltiplicando gli elementi e quando si
cambia colonna si sommi. Andare poi dal basso sottraendo).\\
Il determinante della matrice
\begin{equation*}
  \begin{bmatrix}[1]
    a_{11} & a_{12} & a_{13}\\
    a_{21} & a_{22} & a_{23}\\
    a_{31} & a_{32} & a_{33}
  \end{bmatrix}
\end{equation*}
è dato dalla risoluzione come segue
\begin{center}
  \begin{tikzpicture}[baseline=(A.center), scale=0.5]
    \tikzset{node style ge/.style={circle}}
    \tikzset{bar/.style = {opacity=.3,line width=4 mm,line cap=round,color=#1}}
    \tikzset{plus/.style = {above left,,opacity=1,circle,fill=#1!50}}
    \tikzset{minus/.style = {below left,,opacity=1,circle,fill=#1!50}}

    \matrix (A) [matrix of math nodes, nodes = {node style ge},,column sep=0 mm] 
    {a_{11} & a_{12} & a_{13}\\
      a_{21} & a_{22} & a_{23}\\
      a_{31} & a_{32} & a_{33}\\
      a_{11} & a_{12} & a_{13}\\
      a_{21} & a_{22} & a_{13}\\
    };

    \draw [bar=blue] (A-1-1.north west) node[plus=blue] {$+$} to (A-3-3.south east);
    \draw [bar=blue] (A-2-1.north west) node[plus=blue] {$+$} to (A-4-3.south east);
    \draw [bar=blue] (A-3-1.north west) node[plus=blue] {$+$} to (A-5-3.south east);
    \draw [bar=red]  (A-3-1.south west) node[minus=red] {$-$} to (A-1-3.north east);
    \draw [bar=red]  (A-4-1.south west) node[minus=red] {$-$} to (A-2-3.north east);
    \draw [bar=red]  (A-5-1.south west) node[minus=red] {$-$} to (A-3-3.north east);
  \end{tikzpicture}
\end{center}

\subsection{Rette}\label{subsec:geomanal:retta}
\begin{center}
  \begin{tikzpicture}
    \begin{axis}[xmin=-2,ymin=0,xmax=3,ymax=3]
      \coordinate (O) at (0,0);
      \node[fill=white,circle,inner sep=0pt] (O-label) at ($(O)+(-135:10pt)$) {$O$};
      \addplot[blue,thick] {1/pi*x+1};
    \end{axis}
  \end{tikzpicture}
\end{center}
Le rette sono definite da un'equazione che ha due forme equivalenti:
\begin{equation}\label{eq:retta1}
  y = mx + q\\
\end{equation}
\begin{equation}\label{eq:retta2}
  ax + by + c = 0
\end{equation}
La forma~\eqref{eq:retta1} è chiamata \emph{esplicita}, la forma~\eqref{eq:retta2} è chiamata 
\emph{implicita}. Da queste due forme possiamo evincere che
\begin{equation*}
  m = -\frac{a}{b} \qquad q = -\frac{c}{b}
\end{equation*}
% Reset the counter
\setcounter{equation}{0}

\subsubsection{Retta passante per due punti $P_1(x_1,y_1)$ e $P_2(x_2,y_2)$}
\begin{equation*}
  \frac{y-y_1}{y_2-y_1}=\frac{x-x_1}{x_2-x_1} \qquad x_1\neq x_2 \land y_1\neq y_2
\end{equation*}

\subsubsection{Condizione di parallelismo}
Perché due rette siano parallele, \textbf{il loro coefficiente angolare deve essere uguale}, ovvero
\begin{equation*}
  r_1 \| r_2 \iff m_1=m_2
\end{equation*}

\subsubsection{Condizione di perpendicolarità}
Perché due rette siano perpendicolari, \textbf{il prodotto dei coefficienti angolari deve essere $-1$},
ovvero
\begin{equation*}
  r_1 \perp r_2 \iff m_1m_2 = -1
\end{equation*}

\subsubsection{Retta parallela ad una data e passante per un punto $P(x_P,y_P)$}
\begin{equation*}
  y-y_P = m(x-x_P)
\end{equation*}

\subsubsection{Retta perpendicolare ad una data e passante per un punto $P(x_P,y_P)$}
\begin{equation*}
  y - y_P = -\frac{1}{m} (x - x_P)
\end{equation*}

\subsubsection{Distanza $d$ tra un punto $P(x_P,y_P)$ e una retta}
Sia $r$ una retta in forma esplicita $ax+by+c$ e $P$ un punto del piano, allora la distanza minima
tra le due è
\begin{equation*}
  d = \frac{\abs{ax_P + by_P +c}}{\sqrt{a^2+b^2}}
\end{equation*}

\subsubsection{Coefficiente angolare $m$ di una retta passante per due punti $P_1(x_1,y_1)$ e 
$P_2(x_2,y_2)$}
\begin{equation*}
  m=\frac{y_2-y_1}{x_2-x_1}
\end{equation*}

\subsection{Fasci di Rette}\label{subsec:geomanal:fasciorette}
Un fascio di rette è una combinazione lineare di tutte le rette generabili modificando un solo 
parametro di una quantità costante.

\subsubsection{Fascio di rette a due parametri}
Sceglti appropriati $\alpha$ e $\beta$ si possono generare tutte le rette possibili utilizzando questa
forma
\begin{equation*}
  \alpha(ax + by + c) + \beta(a_1x + b_1y + c_1) = 0
\end{equation*}
\begin{equation*}
  (\alpha a + \beta a_1)x + (\alpha b + \beta b_1)y + \alpha c + \beta c_1 = 0
\end{equation*}

\subsubsection{Fascio di rette ad un parametro}
Questa forma esclude una sola retta, per $k = 0$.
\begin{equation*}
  ax + by + c + k(a_1x + b_1y + c_1) = 0
\end{equation*}
Si noti che $k = \frac{\beta}{\alpha}$

\subsubsection{$k$ avendo una retta del fascio, la retta esclusa e un punto su $a_1x + b_1y + c = 0$}
\begin{center}
  \begin{tikzpicture}
    \begin{axis}[xmin=-3,ymin=-3,xmax=3,ymax=3]
      \coordinate (O) at (0,0);
      \node[fill=white,circle,inner sep=0pt] (O-label) at ($(O)+(-135:10pt)$) {$O$};
      \addplot[thick,dashed,teal]{1.7}node[pos=0.3,above,text width=2cm]{Esclusa: $a_1x+b_1y+c=0$};
      \addplot[thick, brown] {3*x}node[pos=0.45,left]{$r:\,ax+by+c=0$};
      \addplot[thick, red] {-4*x+4};
      \LabelPoint[mark=*,color=blue][right]{1.5}{-2}{$Q(x_Q,y_Q)$}
    \end{axis}
    %\draw[teal, dashed] (0,-0.15) -- (3,-0.15)
    %        node[pos=0, above left]{Esclusa: $a_1x + b_1y + c = 0$};
    %\draw[brown] (0,-3) -- (2,1)
    %        node[pos=0, below]{$r: ax + by + c = 0$};
    %\draw[red] (2.5,-3) -- (1,1)
    %        node[pos = 0.5, left]{$k =?$};
    %\filldraw[blue] (2,-1.65) circle(0.05)
    %        node[right]{$Q(x_Q, y_Q)$};
  \end{tikzpicture}
\end{center}
\begin{equation*}
  \mathcolor{red}{k} = -\frac{
    \mathcolor{brown}{a}\mathcolor{blue}{x_Q} + \mathcolor{brown}{b}\mathcolor{blue}{y_Q} +
    \mathcolor{brown}{c}
    }{
    \mathcolor{teal}{a_1}\mathcolor{blue}{x_Q} + \mathcolor{teal}{b_1}\mathcolor{blue}{y_Q} +
    \mathcolor{teal}{c_1}
  }
\end{equation*}

\subsubsection{Retta di un fascio con coefficiente angolare $m$ passante per un punto $P(x_P,y_P)$}
\begin{equation*}
  y-y_P = m(x-x_P)
\end{equation*}

\subsection{Circonferenza}\label{subsec:geomana:circ}
La circonferenza è una conica i cui punti sono tutti equidistanti dal centro $C$.
\begin{center}
  \begin{tikzpicture}
    \begin{axis}[xmin=-1,ymin=-1,xmax=1,ymax=1]
      \coordinate (O) at (0,0);
      \node[fill=white,circle,inner sep=0pt] (O-label) at ($(O)+(-135:10pt)$) {$O$};
      \draw[thick, blue] (0,0) circle (1);
      \draw[thick, red, ->] (0,0) -- (1,0)
        node[pos=0.5, above]{$r$};
    \end{axis}
  \end{tikzpicture}
\end{center}
Le equazioni delle circonferenze hanno 2 forme
\begin{equation*}
  \mathscr{C}:\,(x-\mathcolor{blue}{x_C})^2 + (y-\mathcolor{blue}{y_C})^2 = \mathcolor{red}{r}^2
\end{equation*}
\begin{equation*}
  \mathscr{C}:\,x^2+y^2+ax+by+c =0
\end{equation*}
Da queste due formule derivano le coordinate del centro
\begin{equation*}
  C\left(-\frac{a}{2},-\frac{b}{2}\right)
\end{equation*}
e la misura del raggio
\begin{equation*}
  \mathcolor{red}{r} = \sqrt{\mathcolor{blue}{x_C}^2+\mathcolor{blue}{y_C}^2-c} = 
  \sqrt{\frac{a^2}{4}+\frac{b^2}{4}-c}
\end{equation*}

\subsubsection{Tangente in $P(x_P,y_P)$}
\begin{equation*}
  x\cdot x_P+y\cdot y_P+a\frac{x+x_P}{2}+b\frac{y+y_P}{2}+c = 0
\end{equation*}

\subsubsection{Area del cerchio}
\begin{equation*}
  \mathscr{A}(\mathscr{C}) = \pi\mathcolor{red}{r}^2
\end{equation*}

\subsubsection{Lunghezza della circonferenza}
\begin{equation*}
  C = 2\pi\mathcolor{red}{r}
\end{equation*}

\subsubsection{Lunghezza dell'arco}
\begin{equation*}
  l = \mathcolor{red}{r}\alpha
\end{equation*}
Si noti che $\alpha$ è in radianti.

\subsubsection{Area del settore}
\begin{equation*}
  \mathscr{A}(\mathscr{S}) = \frac{1}{2}\mathcolor{red}{r}^2\alpha
\end{equation*}
Si noti che $\alpha$ è in radianti.

\subsection{Fasci di circonferenze}\label{subsec:geomanal:fasciocirc}
Un fascio di circonferenze è una combinazione lineare di utte le circonferenze generabili modificando
un parametro di una certa quantità costante.

\subsubsection{Fascio di circonferenze ad un parametro}
Scelti appropriati $\alpha$ e $\beta$ si possono generare tute le circonferenze possibili utilizzando 
questa forma
\begin{equation*}
  \alpha(x^2+y^2+a_1x+b_1y+c_1) + \beta(x^2+y^2+a_2x+b_2y+c_2) = 0
\end{equation*}
\begin{equation*}
  (\alpha+\beta)x^2+(\alpha+\beta)y^2+(\alpha a_1+\beta a_2)x+(\alpha b_1+\beta b_2)y+
  \alpha c_1+\beta c_2 = 0
\end{equation*}

\subsubsection{Fascio di circonferenze a due parametri}
Questa forma esclude una circonferenza per $k=0$.
\begin{equation*}
  x^2+y^2+ax+by+c+k(x^2+y^2+a_1x+b_1y+c_1) = 0
\end{equation*}
Si noti che $k=\frac{\alpha}{\beta}$

\subsection{Parabola}\label{subsec:geomanal:parabola}
\begin{center}
  \begin{tikzpicture}
    \begin{axis}[xmin=-2,ymin=-2,xmax=4,ymax=4]
      \coordinate (O) at (0,0);
      \node[fill=white,circle,inner sep=0pt] (O-label) at ($(O)+(-135:10pt)$) {$O$};
      \addplot[red,thick,samples=1000] {x^2};
      \addplot[blue,thick,samples=1000] (x*x,x);
    \end{axis}

    \begin{scope}[shift={(2,-4)}]
      \draw (0,0) -- ++(2,0);
      \draw[dashed] (1,3) -- ++(0,-4);
      \draw[red,thick] plot[domain=0:2] (\x, {(\x-1)^2+1});
      \filldraw[red] (1,1)
        node[below left]{$V$};
      \filldraw[red] (1,1.5)
        node[above right]{$F$};
    \end{scope}
  \end{tikzpicture}
\end{center}
Una parabola può essere descritta con l'asse focale parallelo all'asse $x$ o all'asse $y$.
\begin{equation*}
  \mathcolor{red}{\mathscr{P}}:\,y=ax^2+bx+c
\end{equation*}
\begin{equation*}
  \mathcolor{blue}{\mathscr{P}}:\,x=ay^2+by+c
\end{equation*}
La direttrice di una parabola è quella che ne da l'inclinazione ed è perpendicolare all'asse di
simmetria.\\
Il vertice di una parabola è il punto più vicino alla direttrice.\\
Il fuoco è il punto la cui distanza da qualsiasi punto della parabola è pari a quella della proiezione
sulla direttrice del punto stesso.

\subsubsection{Elementi di una parabola con asse focale parallelo a $x$}
\begin{center}
  \begin{tabular}{c | c}
    \textbf{Fuoco} & $\left(-\dfrac{b}{2a},\dfrac{1-\Delta}{4a}\right)$\\\hline
    \textbf{Vertice} & $\left(-\dfrac{b}{2a}, -\dfrac{\Delta}{4a}\right)$\\\hline
    \textbf{Direttrice} & $y=-\dfrac{1+\Delta}{4a}$\\\hline
    \textbf{Asse di simmetria} & $x=-\dfrac{b}{2a}$\\\hline
    \textbf{Tangente in un punto} & $\dfrac{y+y_0}{2}=axx_0+b\dfrac{x+x_0}{2}+c$
  \end{tabular}
\end{center}

\subsubsection{Elementi di una parabola con asse focale parallelo a $y$}
\begin{center}
  \begin{tabular}{c | c}
    \textbf{Fuoco} & $\left(\dfrac{1-\Delta}{4a},-\dfrac{b}{2a}\right)$\\\hline
    \textbf{Vertice} & $\left(-\dfrac{\Delta}{4a},-\dfrac{b}{2a}\right)$\\\hline
    \textbf{Direttrice} & $x=-\dfrac{1+\Delta}{4a}$\\\hline
    \textbf{Asse di simmetria} & $y=-\dfrac{b}{2a}$\\\hline
    \textbf{Tangente in un punto} & $\dfrac{x+x_0}{2}=ayy_0+b\dfrac{y+y	_0}{2}+c$
  \end{tabular}
\end{center}

\subsubsection{Parabole di vertice $V(x_V,y_V)$}
\begin{equation*}
  y-y_V=a(x-x_V)^2
\end{equation*}

\subsubsection{Area di un segmento parabolico}
\begin{center}
  \begin{tikzpicture}
    \begin{axis}[
      xmin=-1.5,
      ymin=-1,
      xmax=2.5,
      ymax=6,
      % to be able to draw the orange filling on another layer
      set layers,
      ]

      % draw the function and the "intersection lines"
      % (please note that I have changed the number of samples and added
      %  the option `smooth' to avoid some numerical instabilities for `f')
      \addplot [name path=f,red,thick,samples=49,smooth] {x^2};
      \addplot [name path=l1,thick] {1/3*x+1.5};
      \addplot [name path=l2,thick] {1/3*x};
      \addplot [name path=p1,thick,green] {-3*x-2.1};
      \addplot [name path=p2,thick,green] {-3*x+6.2};

      % find the intersection points of the black and green lines
      \path [
        name intersections={
          of=l1 and p1,
          by={A},
        },
        ];
      \path [
        name intersections={
          of=l1 and p2,
          by={B},
        },
        ];
      \path [
        name intersections={
          of=l2 and p2,
          by={C},
        },
        ];
      \path [
        name intersections={
          of=l2 and p1,
          by={H},
        },
        ];

      % create coordinate at origin
      \coordinate (O) at (0,0);

      % create invisible clip paths which are needed for the orange filling
      \path [name path=clippath1] (A) -- (H) -- (C) -- cycle;
      \path [name path=clippath2] (O) -- (C) -- (B) -- cycle;

      % draw the intersection points
      \pgfmathsetlengthmacro{\Radius}{2pt}
      \fill
      (A) circle (\Radius)
      (B) circle (\Radius)
      (C) circle (\Radius)
      (H) circle (\Radius)
      ;

      % label the intersection points
      \node [coordinate,label=below right:$O$] at (O) {};
      \node [coordinate,label=above right:$A$] at (A) {};
      \node [coordinate,label=above right:$B$] at (B) {};
      \node [coordinate,label=below right:$C$] at (C) {};
      \node [coordinate,label=below left:$H$] at (H) {};

      % fill the area between the intersection points on a lower layer
      % so the red function line doesn't have to be plotted twice
      \begin{pgfonlayer}{axis ticks}
        % left half
        \fill [
        orange,
        fill opacity=0.5,
        % (this is the TikZ equivalent to PGFPlots `fill between')
        intersection segments={
          of=f and clippath1,
          % (here we can draw -- in general -- an arbitrary path
          %  between the path elements of intersection points.
          %  Of course here we want to find the path that surrounds
          %  the area that we want to fill.)
          sequence={R1[reverse] -- L2},
        },
        ];
        % right half
        \fill [
        orange,
        fill opacity=0.5,
        intersection segments={
          of=f and clippath2,
          sequence={R{-2} -- L{-2}[reverse]},
        },
        ];
      \end{pgfonlayer}

      % draw the blue filling
      \addplot [
      fill=none,
      ] fill between [
      of=f and l1,
      split,
      every segment no 1/.style={
        fill=blue,
        fill opacity=0.5,
      },
      ];

      %        % ---------------------------------------------------------------------
      %        % for debugging purpose only
      %        % ---------------------------------------------------------------------
      %        % To find the right `sequence' you can play with the elements.
      %        % Just start with one single element like `R1' to see what happens and
      %        % then replace them until you found the right ones and connect them in
      %        % the right order.
      %        \draw [
      %            blue,
      %            very thin,
      %            |->,
      %            intersection segments={
      %                of=f and clippath1,
      %                sequence={
      %                    % Because we know that the "green/black" line is needed
      %                    % from the start to the first intersection point, for sure
      %                    % we need `R1'.
      %                    % And we also know that we need for the "red" line the part
      %                    % from the first (not real) intersection point above (left)
      %                    % of point A (crossing of the green and red line) to the
      %                    % second intersection point (at point O)
      %                    % (There is still some magic left why there is this "not
      %                    %  real" intersection point ...)
      %                    R1[reverse] -- L2
      %%                    % so the reverse path is also fine, which can be done by
      %%                    % reversing the "pathes" ...
      %%                    R1 -- L2[reverse]
      %%                    % ... or the elements of the pathes which offers another
      %%                    % two possibilities to do this
      %%                    L2 -- R1[reverse]
      %%                    L2[reverse] -- R1
      %%                    % Another possibility to avoid the `[reverse] you could
      %%                    % simply reverse the path directly `clippath1' from
      %%                    %     (A) -- (H) -- (C) -- cycle
      %%                    % to
      %%                    %     (C) -- (H) -- (A) -- cycle
      %%                    % in the (above) definition of that path.
      %%                    % Can you imagine how the right elements and sequence is then?
      %%                    % (One tip: It is not as simple as `L2 -- R1')
      %                },
      %            },
      %        ];
      %        \draw [
      %            blue,
      %            very thin,
      %            |->,
      %            intersection segments={
      %                of=f and clippath2,
      %                sequence={
      %%                    % try to find the right elements and orders here yourself
      %                    R{-1}
      %                },
      %            },
      %        ];
      %        % ---------------------------------------------------------------------

    \end{axis}
  \end{tikzpicture}
\end{center}
\begin{equation*}
  \mathscr{A}(\mathscr{F}) = \frac{2}{3}\overline{AB}\cdot\overline{AH}
\end{equation*}
E ovviamente l'area esterna alla curva sarebbe
\begin{equation*}
  \mathscr{A}(\mathscr{F}') = \frac{1}{3}\overline{AB}\cdot\overline{AH}
\end{equation*}

\subsubsection{Formule di sdoppiamento}
Le formule di sdoppiamento servono per determinare le tangenti in un punto $P(x_0,y_0)$.\\
Se $d\|y$
\begin{equation*}
  \frac{y+y_0}{2}=axx_0+b\frac{x+x_0}{2}+c
\end{equation*}
Se $d\|x$
\begin{equation*}
  \frac{x+x_0}{2}=ayy_0+b\frac{y+y_0}{2}+c
\end{equation*}

\subsubsection{Coefficiente angolare della tangente}
\begin{equation*}
  m = \frac{1}{2ay_0+b} = 2ax_0+b
\end{equation*}

\subsection{Ellisse}\label{subsec:geomanal:ellisse}
\begin{center}
  \begin{tikzpicture}
    \begin{axis}[xmin=-2,ymin=-2,xmax=2,ymax=2]
      \coordinate (O) at (0,0);
      \node[fill=white,circle,inner sep=0pt] (O-label) at ($(O)+(-135:10pt)$) {$O$};
      \draw[thick,red] (axis cs:0,0) ellipse[x radius=2, y radius=1];
      \LabelPoint[mark=*][above]{-1}{0}{$F_1$}
      \LabelPoint[mark=*][above]{1}{0}{$F_2$}
    \end{axis}
  \end{tikzpicture}
\end{center}
Un'ellissi ha due assi, uno maggiore uno minore. Le loro semi-lunghezze (quindi i semi-assi) si 
denominano $a$ (che contiene i fuochi) e $b$.\\
I fuochi sono i due punti tali che preso un punto $P\in\mathscr{E}$, 
$\overline{PF_1}+\overline{PF_2} = 2a$.
\begin{equation*}
  \mathscr{E}:\,\frac{x^2}{a^2}+\frac{y^2}{b^2}=1
\end{equation*}
Tra i semi-assi vige la seguente proprietà
\begin{equation*}
  a^2-c^2=b^2
\end{equation*}
e quindi
\begin{equation*}
  c = \begin{cases}
    a^2-b^2,\, &\text{se } a > b\\
    b^2-a^2,\, &\text{se } a < b
  \end{cases}
\end{equation*}

\subsubsection{Eccentricità}
L'eccentricità è lo schiacciamento dell'ellisse sull'asse maggiore. È un valore compreso tra $0$ e 
$1$.\\
Se $a>b$
\begin{equation*}
  e = \frac{c}{a} = \frac{\sqrt{a^2-b^2}}{a}=\sqrt{1-\frac{b^2}{a^2}}
\end{equation*}
Se $b>a$
\begin{equation*}
  e = \frac{c}{b} = \sqrt{1-\frac{a^2}{b^2}}
\end{equation*}

\subsubsection{Area dell'ellisse}
\begin{equation*}
  \mathscr{A}(\mathscr{E}) = ab\pi
\end{equation*}

\subsubsection{Tangenti all'ellisse}
Per trovare la tangente all'esllise abbiamo due modi:
\begin{equation*}
  \frac{xx_0}{a^2}+\frac{yy_0}{b^2}=1
\end{equation*}
oppure fare il sistema tra la retta generica per $P$ e fare in modo che il discriminante si annulli:
\begin{align*}
  \begin{dcases}
    \frac{x^2}{a^2}+\frac{y^2}{b^2}=1\\
    y-y_0=m(x-x_0)
  \end{dcases}\rightarrow\\ \frac{\Delta}{4}=a^4m^2q^2-a^2(q^2-b^2)(b^2+a^2m^2) = 0
\end{align*}
Il vantaggio di questo secondo metodo è che può anche trovare le rette secanti ed esterne all'ellisse
(rispettivamente con $\dfrac{\Delta}{4}>0$ e $\dfrac{\Delta}{4}<0$). È sicuramente più laborioso e
difficile da ricordare.

\subsection{Iperbole}
\begin{center}
  \begin{tikzpicture}
    \begin{axis}[xmin=-5,ymin=-5,xmax=5,ymax=5]
      \coordinate (O) at (0,0);
      \node[fill=white,circle,inner sep=0pt] (O-label) at ($(O)+(-135:10pt)$) {$O$};
      \addplot[red,thick,domain=-2:2] ({cosh(x)}, {sinh(x)});
      \addplot[red,thick,domain=-2:2] ({-cosh(x)}, {sinh(x)});
      \addplot[red,dashed] expression {x};
      \addplot[red,dashed] expression {-x};
    \end{axis}
  \end{tikzpicture}
\end{center}
L'iperbole può essere descritta in più modi.\\
I fuochi sono i due punti tali che per un punto $P\in\mathscr{I}$, 
$\lvert \overline{PF_1}-\overline{PF_1}\rvert=2a$.\\
L'equazione dell'iperbole con i fuochi su $x$ è
\begin{equation*}
  \mathscr{I}:\,\frac{x^2}{a^2}-\frac{y^2}{b^2}=1
\end{equation*}
Quella con i fuochi su $y$ è
\begin{equation*}
  \mathscr{I}:\,\frac{x^2}{a^2}-\frac{y^2}{b^2}=-1
\end{equation*}
Tra i parametri $a$ e $b$ vige che $a < c$ e $c^2 = a^2+b^2$.

\subsubsection{Asintoti}
Gli asintoti sono le rette che l'iperbole tende a raggiungere senza mai toccare
\begin{equation*}
  y=\pm\frac{b}{a}x
\end{equation*}

\subsubsection{Eccentricità}
L'eccentricità dell'iperbole è il rapporto
\begin{equation*}
  e=\frac{c}{a}=\sqrt{1+\frac{b^2}{a^2}}
\end{equation*}
se l'iperbole ha i fuochi su $x$,
\begin{equation*}
  e = \frac{c}{b}=\sqrt{1+\frac{a^2}{b^2}}
\end{equation*}
altrimenti. Si noti anche che $e > 1$ per ogni iperbole.

\subsubsection{Iperbole equilatera}
Se $a=b$, l'iperbole si definisce equilatera e le equazioni diventano 
\begin{equation*}
  x^2-y^2=a^2
\end{equation*}
se $F\in x$,
\begin{equation*}
  y^2-x^2=a^2
\end{equation*}
se $F\in y$.\\
Questo comporta che $c = a\sqrt{2}$ e che $e=\sqrt{2}$.\\\\
Può anche essere descritta l'iperbole da
\begin{equation*}
  xy=k
\end{equation*}

\subsubsection{Formule di sdoppiamento}
Vengono ora riportate le formule di sdoppiamento che cambiano in base all'equazione dell'iperbole
\begin{center}
  \begin{tabular}{c|c}
    Equazione & Tangente\\\hline
    $\dfrac{x^2}{a^2}-\dfrac{y^2}{b^2}=1$ & $\dfrac{xx_0}{a^2}-\dfrac{yy_0}{b^2}=1$\\\hline
    $\dfrac{x^2}{a^2}-\dfrac{y^2}{b^2}=-1$ & $\dfrac{xx_0}{a^2}-\dfrac{yy_0}{b^2}=-1$\\\hline
    $x^2-y^2=a^2$ & $xx_0-yy_0=a^2$\\\hline
    $x^2-y^2=-a^2$ & $xx_0-yy_0=-a^2$\\\hline
    $xy=k$ & $xx_0\cdot yy_0=k$
  \end{tabular}
\end{center}

\subsubsection{Iperbole equilatera traslata}
Si trova molto spesso una versione traslata di un'iperbole. Questa è la sua generale forma
\begin{equation*}
  y=\frac{ax+b}{cx+d}
\end{equation*}
E gli asintoti sono
\begin{equation*}
  x=-\frac{d}{c}\qquad y=\frac{a}{c}
\end{equation*}
con il centro di simmetria
\begin{equation*}
  O\left(-\frac{d}{c},\frac{a}{c}\right)
\end{equation*}
