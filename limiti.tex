%!TEX ROOT=formularioMatematica.tex

\section{Limiti}\label{sec:limiti}
Per introdurre il concetto di limite, prendiamo ad esempio la funzione
\begin{equation*}
  f:\,\mathscr{D}_f\mapsto\mathbb{R}\mid x\mapsto\frac{2x^2-8}{x-2}
\end{equation*}
essendo
\begin{equation*}
  \mathscr{D}_f=\mathbb{R}-\{2\}
\end{equation*}
La funzione non è definita per $x = 2$ però possiamo comunque trovare i valori della funzione per numeri
che si avvicinano sempre più a $2$
\begin{center}
  \begin{tabular}{cc}
    $\boldsymbol{x}$ & $\boldsymbol{f(x)}$\\\hline
    $1$ & $6$\\
    $1.5$ & $7$\\
    $1.9$ & $7.8$\\
    $1.9991$ & $7.9982$\\
    $\ldots$ & $\ldots$\\
    $2.0001$ & $8.002$\\
    $2.1$ & $8.2$\\
    $2.5$ & $9$\\
    $3$ & $10$ 
  \end{tabular}
\end{center}
Come notiamo dalla tabella, più i ci si avvicina a $2$ più i valori si avvicinano a $8$. A questo
comportamento diamo il nome di \textbf{limite finito}.\\
Possiamo quindi dire che per un numero $\varepsilon$ positivo
\begin{equation*}
  \left\lvert\frac{2x^2-8}{x-2}-8\right\rvert<\varepsilon
\end{equation*}
Questa disequazione ammette come soluzioni un intervalo opportuno di centro $x=2$. Tenuto conto che
\begin{align*}
  &2x^2-8=2(x-2)(x+2)
  \intertext{riducendo}
  &\abs{2x-4}<\varepsilon\quad x \neq 2
  \intertext{ovvero}
  &-\varepsilon<2x-4<\varepsilon\quad x \neq 2
  \intertext{ossia}
  &2-\frac{\varepsilon}{2}<x<2+\frac{\varepsilon}{2}\quad x \neq 2
  \intertext{Questo vuol dire che se}
  &x\in{\left]{2-\frac{\varepsilon}{2}},{2+\frac{\varepsilon}{2}}\right[}\quad x \neq 2
  \intertext{i corrispondenti valori di $f(x)$ distano da $8$ meno di $\varepsilon$}
\end{align*}
Scriveremo allora
\begin{equation*}
  \lim\limits_{x\to2}\frac{2x^2-8}{x-2}=8
\end{equation*}
che si legge \emph{il limite per $x$ che tende a $2$ di $\frac{2x^2-8}{x-2}$ è uguale a $8$}.\\\\
Consideriamo ora la funzione
\begin{equation*}
  f:\,\mathscr{D}_f\mapsto\mathbb{R}\mid x\mapsto\frac{1}{(x+1)^2}
\end{equation*}
essendo
\begin{equation*}
  \mathscr{D}_f = \mathbb{R}-\{-1\}
\end{equation*}
Attribuiamo ora a $x$ valori sempre più vicini a $-1$
\begin{center}
  \begin{tabular}{cc}
    $\boldsymbol{x}$ & $\boldsymbol{f(x)}$\\\hline
    $-2$ & $1$\\
    $-1.5$ & $4$\\
    $-1.001$ & $1,000,000$\\
    $\ldots$ & $\ldots$\\
    $0$ & $1$\\
    $-0.5$ & $4$\\
    $-0.99995$ & $400,000,000$
  \end{tabular}
\end{center}
Notiamo che per valori che si avvicinano a $-1$ otteniamo sempre valori molto grandi. A questo 
comportamento si da il nome di \textbf{limite a più infinito ($+\infty$)}.\\
Quindi si può scrivere
\begin{equation*}
  \lim\limits_{x\to-1}\frac{1}{(x+1)^2}=+\infty
\end{equation*}
che sgnifica che comunque si prenda un numero $M$ la disuguaglianza
\begin{equation*}
  \frac{1}{(x+1)^2}>M
\end{equation*}
è soddisfatta dai punti di un intorno di $-1$, escluso $-1$ stesso.\\
Supposto che $x\neq-1$ l'equazione equivale a
\begin{align*}
  &(x+1)^2<\frac{1}{M}
  \intertext{verificata per}
  &-\frac{1}{\sqrt{M}}<x+1<\frac{1}{\sqrt{M}}
  \intertext{cioè}
  &-1-\frac{1}{\sqrt{M}}<x<-1+\frac{1}{\sqrt{M}}\quad x\neq-1
\end{align*}
Tali valori effettivamente rappresentano un intorno di $-1$ escluso $-1$ stesso.\\
Analogamente al limite che tende a $+\infty$, si può trovare il limite a $-\infty$.\\\\
Un problema simile a quelli precedenti è quello di un valore che dopo un po' si stabilizza. In altre
parole, un valore che tendendo ad $\infty$ tende ad un numero finito. Questi sono definiti \textbf{
limiti finiti di una funzione all'infinito}. In simboli
\begin{equation*}
  \lim\limits_{x\to+\infty}f(x)=l
\end{equation*}
Possiamo ulteriormente estendere il concetto a \textbf{limiti infiniti di una funzione all'infinito}.
Ad esempio
\begin{equation*}
  \lim\limits_{x\to-\infty}f(x)=+\infty
\end{equation*}


\subsection{Definizione di limite finito}
\begin{definizioneLimiteFinito}
  Sia $f$ una funzione definita in un intorno $I$ del punto $x_0$, senza che sia necessariamente
  definita in $x_0$.\\
  Si dice che il numero $l$ è il \textbf{limite} della funzione $f$ nel punto $x_0$ e si scrive
  \begin{equation*}
    \lim\limits_{x\to x_0}f(x)=l
  \end{equation*}
  se, fissato comunque un numero $\varepsilon>0$, è possibile determinare in corrispondenza di esso 
  un numero $\delta_\varepsilon>0$ tale che, per ogni $x$ appartenente a $I$ verificante la 
  condizione
  \begin{equation*}
    0<\abs{x-x_0}<\delta_\varepsilon
  \end{equation*}
  risulti
  \begin{equation*}
    \abs{f(x)-l}<\varepsilon
  \end{equation*}
  In simboli
  \begin{align*}
    \lim\limits_{x\to x_0} f(x) &= l\\
    \ArrowBetweenLines
    \forall \varepsilon, \exists\,\delta_\varepsilon>0\mid\forall x: 0<\vert x-x_0\vert 
    &<\delta_\varepsilon \Rightarrow \vert f(x) - l\vert < \varepsilon
  \end{align*}
\end{definizioneLimiteFinito}

\subsection{Definizione di limite infinito}
\subsubsection{A $+\infty$}
\begin{definizioneLimiteInfinito1}
  Sia $f$ una funzione definita in un intorno $I$ di $x_0$, escluso al più il punto $x_0$. Si dice
  che
  \begin{equation*}
    \lim\limits_{x\to x_0}f(x)=+\infty
  \end{equation*}
  se, fissato comunque un numero $M$, è possibile determinare in corrispondenza di esso un numero
  $\delta_M>0$ tale che, per ogni $x$ di $I$ verificante la condizione
  \begin{equation*}
    0<\abs{x-x_0}<\delta_M
  \end{equation*}
  risulti
  \begin{equation*}
    f(x)>M
  \end{equation*}
  In simboli
  \begin{align*}
    \lim\limits_{x\to x_0} f(x) &= l\\
    \ArrowBetweenLines
    \forall M>0, \exists\,\delta_M>0\mid\forall x: 0<\vert x-&x_0\vert < \delta_M \Rightarrow
    f(x) > M
  \end{align*}
\end{definizioneLimiteInfinito1}
\subsubsection{A $-\infty$}
\begin{definizioneLimiteInfinito2}
  Sia $f$ una funzione definita in un intorno $I$ di $x_0$, escluso al più il punto $x_0$. Si dice
  che
  \begin{equation*}
    \lim\limits_{x\to x_0}f(x)=-\infty
  \end{equation*}
  se, fissato comunque un numero $M$, è possibile determinare in corrispondenza di esso un numero
  $\delta_M>0$ tale che, per ogni $x$ di $I$ verificante la condizione
  \begin{equation*}
    0<\abs{x-x_0}<\delta_M
  \end{equation*}
  risulti
  \begin{equation*}
    f(x)<M
  \end{equation*}
  In simboli
  \begin{align*}
    \lim\limits_{x\to x_0} f(x) &= l\\
    \ArrowBetweenLines
    \forall M, \exists\,\delta_M>0\mid\forall x: 0<\vert x-x_0\vert &< \delta_M \Rightarrow
    f(x) < M
  \end{align*}
\end{definizioneLimiteInfinito2}

\subsection{Definizione di limite finito di una funzione all'infinito}
\subsubsection{Per $x\to+\infty$}
\begin{definizioneLimiteInfinitoFinito1}
  Sia $f$ una funzione definita in un insieme $\mathscr{D}_f$ illimitato superiormente.\\
  Si dice che
  \begin{equation*}
    \lim\limits_{x\to+\infty}f(x)=l
  \end{equation*}
  se, fissato comunque un numero $\varepsilon>0$ è possibile determinare in corrispondenza di esso 
  un numero $k_\varepsilon$ tale che, per ogni $x\in\mathscr{D}_f$ e maggiore di $k_\varepsilon$, 
  risulti
  \begin{equation*}
    \abs{f(x)-l}<\varepsilon
  \end{equation*}
  In simboli
  \begin{align*}
    \lim\limits_{x\to x_0} f(x) &= l\\
    \ArrowBetweenLines
    \forall \varepsilon, \exists\,k_\varepsilon>0\mid\forall x: x>&k_\varepsilon \Rightarrow
    \vert f(x) - l\vert < \varepsilon
  \end{align*}
\end{definizioneLimiteInfinitoFinito1}
\subsubsection{Per $x\to-\infty$}
\begin{definizioneLimiteInfinitoFinito2}
  Sia $f$ una funzione definita in un insieme $\mathscr{D}_f$ illimitato superiormente.\\
  Si dice che
  \begin{equation*}
    \lim\limits_{x\to-\infty}f(x)=l
  \end{equation*}
  se, fissato comunque un numero $\varepsilon>0$ è possibile determinare in corrispondenza di esso 
  un numero $k_\varepsilon$ tale che, per ogni $x\in\mathscr{D}_f$ e minore di $k_\varepsilon$, 
  risulti
  \begin{equation*}
    \abs{f(x)-l}<\varepsilon
  \end{equation*}
  In simboli
  \begin{align*}
    \lim\limits_{x\to x_0} f(x) &= l\\
    \ArrowBetweenLines
    \forall \varepsilon, \exists\,k_\varepsilon>0\mid\forall x: x<&k_\varepsilon \Rightarrow
    \vert f(x) - l\vert < \varepsilon
  \end{align*}
\end{definizioneLimiteInfinitoFinito2}

\subsection{Definizione di limite infinito di una funzione all'infinito}
\subsubsection{A $+\infty$}
\begin{definizioneLimiteInfinitoInfinito1}
  Sia $f$ una funzione definita in un insieme $\mathscr{D}_f$ illimitato superiormente 
  [inferiormente]. Si dice che
  \begin{equation*}
    \lim\limits_{\substack{x\to+\infty\\ [x\to-\infty]}}f(x)=+\infty
  \end{equation*}
  se, fissato comunque un numero $M$, è possibile determinare in corrispondenza di esso un numero 
  $k_M$ tale che, per ogni $x\in\mathscr{D}_f$ che verifichi la condizione $x>k_M\,[x<k_M]$, risulti
  \begin{equation*}
    f(x)>M
  \end{equation*}
  In simboli
  \begin{align*}
    \lim\limits_{\substack{x\to+\infty\\ [x\to-\infty]}}f(x)&=+\infty\\
    \ArrowBetweenLines
    \forall k_M>0, \exists\,M>0\mid\forall x: x>&k_M[<k_M] \Rightarrow f(x) > M
  \end{align*}
\end{definizioneLimiteInfinitoInfinito1}
\subsubsection{A $-\infty$}
\begin{definizioneLimiteInfinitoInfinito2}
  Sia $f$ una funzione definita in un insieme $\mathscr{D}_f$ illimitato superiormente 
  [inferiormente]. Si dice che
  \begin{equation*}
    \lim\limits_{\substack{x\to+\infty\\ [x\to-\infty]}} f(x)=-\infty
  \end{equation*}
  se, fissato comunque un numero $M$, è possibile determinare in corrispondenza di esso un numero 
  $k_M$ tale che, per ogni $x\in\mathscr{D}_f$ che verifichi la condizione $x>k_M\,[x<k_M]$, risulti
  \begin{equation*}
    f(x)<M
  \end{equation*}
  In simboli
  \begin{align*}
    \lim\limits_{\substack{x\to+\infty\\ [x\to-\infty]}}f(x)&=+\infty\\
    \ArrowBetweenLines
    \forall k_M, \exists\,M>0\mid\forall x: x>k_M&[<k_M] \Rightarrow f(x) < M
  \end{align*}
\end{definizioneLimiteInfinitoInfinito2}

\subsection{Limite sinistro e destro}
Avere limite $l$ in un punto $x_0$ significa per una funzione essere regolare, ovvero assumere valori
sempre più prossimi a $l$ tanto $x$ è prossimo a $x_0$.\\
Questa regolarità però può mancare in senso assoluto. Ciò avviene quando la funzione si stabilizza
su due numeri diversi a seconda che ci si avvicini da destra o da sinistra.
\begin{equation*}
  \lim\limits_{x\to x_0^+}f(x)
\end{equation*}
indica un limite destro,
\begin{equation*}
  \lim\limits_{x\to x_0^-}f(x)
\end{equation*}
un limite sinistro.
\begin{definizioneLimiteFinitoDestro}
  Sia $f$ una funzione definita in un intorno destro $I^+(x_0)$ di $x_0$, privato al più del punto
  $x_0$.\\
  Si dice che
  \begin{equation*}
    \lim\limits_{x\to x_0^+}f(x) = l
  \end{equation*}
  se, fissato comunque un numero $\varepsilon>0$, è possibile determinare in corrispondenza di esso
  un numero $\delta_\varepsilon>0$ tale che, per ogni $x\in I^+(x_0)$ verificante la condizione
  \begin{equation*}
    0<x-x_0<\delta_\varepsilon
  \end{equation*}
  risulti
  \begin{equation*}
    \abs{f(x)-l}<\varepsilon
  \end{equation*}
  In simboli
  \begin{align*}
    \lim\limits_{x\to x_0^+} f(x) &= l\\
    \ArrowBetweenLines
    \forall \varepsilon, \exists\,\delta_\varepsilon>0\mid\forall x: x_0< x <x_0+&\delta_\varepsilon 
    \Rightarrow \vert f(x) - l\vert < \varepsilon
  \end{align*}
\end{definizioneLimiteFinitoDestro}
\begin{definizioneLimiteFinitoSinistro}
  Sia $f$ una funzione definita in un intorno sinistro $I^-(x_0)$ di $x_0$, privato al più del punto
  $x_0$.\\
  Si dice che
  \begin{equation*}
    \lim\limits_{x\to x_0^-}f(x) = l
  \end{equation*}
  se, fissato comunque un numero $\varepsilon>0$, è possibile determinare in corrispondenza di esso
  un numero $\delta_\varepsilon>0$ tale che, per ogni $x\in I^-(x_0)$ verificante la condizione
  \begin{equation*}
    0<x-x_0<\delta_\varepsilon
  \end{equation*}
  risulti
  \begin{equation*}
    \abs{f(x)-l}<\varepsilon
  \end{equation*}
  In simboli
  \begin{align*}
    \lim\limits_{x\to x_0^-} f(x) &= l\\
    \ArrowBetweenLines
    \forall \varepsilon, \exists\,\delta_\varepsilon>0\mid\forall x: x_0-\delta_\varepsilon<x&<x_0
    \Rightarrow \vert f(x) - l\vert < \varepsilon
  \end{align*}
\end{definizioneLimiteFinitoSinistro}

\subsection{Definizione generale di limite}
Fin'ora abbiamo elencato varie definizioni formali ma ce n'è una generale, che le comprenda tutte? 
Certo che sì e anzi, è anche più facile da ricordare in quanto è una unica. Sapendo poi adattarla, si
ricavano tute le altre.
\begin{definizioneGeneraleLimite}
  Siano $V(l)$ e $U(x_0)$ due intorni dei rispettivi parametri. Si ha allora che
  \begin{align*}
    \lim\limits_{x\to x_0} f(x) &= l\\
    \ArrowBetweenLines
    \forall V(l), \exists\,U(x_0)\mid\forall x\in U(x_0)&\setminus\{x_0\}\Rightarrow f(x)\in V(l)
  \end{align*}
\end{definizioneGeneraleLimite}
Questo permette di imparare una sola formula che però, opportunamente adattata, permette di ricavare
le definizioni formali di ogni limite.

\subsection{Teoremi sui limiti}
\begin{uniLim}\hypertarget{teor:uniLim}{}
  Se una funzione ammette limite per $x\to x_0$, tale limite è unico.
\end{uniLim}
\begin{confrontoLim}\hypertarget{teor:confLim}{}
  Siano $f$, $g$ e $h$ tre funzioni definite in un intorno $I$ di $x_0$, escluso al più $x_0$, e tali
  che per ogni $x\in I$ risulti
  \begin{equation*}
    f(x)\leq g(x)\leq h(x)
  \end{equation*}
  Se
  \begin{equation*}
    \lim\limits_{x\to x_0} f(x) = \lim\limits_{x\to x_0} h(x) = l
  \end{equation*}
  allora risulterà
  \begin{equation*}
    \lim\limits_{x\to x_0}g(x)=l
  \end{equation*}
\end{confrontoLim}
\begin{permanenzaSegno}\hypertarget{teor:segno}{}
  Se
  \begin{equation*}
    \lim\limits_{x\to x_0}f(x)=l\neq0
  \end{equation*}
  esiste un intorno $I(x_0)$, privato al più del punto $x_0$, in cui la funzione assume lo stesso 
  segno di $l$.\\
  Viceversa, se esiste un intorno $I(x_0)$ di $x_0$ privato al più di $x_0$, in cui risulta $f(x)>0$
  [$f(x)<0$], e se esiste $\lim\limits_{x\to x_0}f(x)=l$ si avrà
  \begin{equation*}
    l\geq0\quad[l\leq0]
  \end{equation*}
\end{permanenzaSegno}

\subsection{Operazioni sui limiti}
\subsubsection{Somma}
\begin{sommaLimiti}\hypertarget{teor:sommaLimiti}{}
  Il limite di una somma di funzioni è uguale alla somma dei limiti se questi sono finiti.
  \begin{equation*}
    \lim\limits_{x\to x_0}[f(x)+g(x)] = l_1+l_2
  \end{equation*}
\end{sommaLimiti}

\subsubsection{Prodotto}
\begin{prodottoLimiti}\hypertarget{teor:prodottoLimiti}{}
  Il limite di un prodotto di funzioni è uguale al prodotto dei limiti delle funzioni se questi sono
  finiti.
  \begin{equation*}
    \lim\limits_{x\to x_0}f(x)\cdot g(x)=l_1\cdot l_2
  \end{equation*}
\end{prodottoLimiti}
Dal prodotto si possono ricavare anche i seguenti 2 teoremi
\begin{prodottoLimiti1}
  Se $f(x)$ è una funzione che ammette limite $l$ per $x$ che tende a $x_0$ e $k$ è un numero reale,
  si ha
  \begin{equation*}
    \lim\limits_{x\to x_0}k\,f(x) = k\cdot l
  \end{equation*}
\end{prodottoLimiti1}
\begin{prodottoLimiti2}
  Se $f(x)$ e $g(x)$ sono due funzioni che per $x$ che tende a $x_0$ hanno limiti $l_1$ e $l_2$, e
  $\lambda$ e $\mu$ sono due numeri reali, si ha
  \begin{equation*}
    \lim\limits_{x\to x_0}[\lambda f(x)+\mu g(x)]=\lambda l_1+\mu l_2
  \end{equation*}
\end{prodottoLimiti2}

\subsubsection{Quoziente}
\begin{quozienteLimiti}
  Se $f(x)$ e $g(x)$ sono due funzioni aventi rispettivamente i limiti $l_1$ e $l_2$ per $x$ che 
  tende a $x_0$ e se $l_2\neq0$ si ha
  \begin{equation*}
    \lim\limits_{x\to x_0}\frac{f(x)}{g(x)}=\frac{l_1}{l_2}
  \end{equation*}
\end{quozienteLimiti}

\subsubsection{Potenza}
\begin{potenzaLimiti}
  Se $\lim\limits_{x\to x_0}f(x)=l$ e $a\in\mathbb{R}_0^+$
  \begin{equation*}
    \lim\limits_{x\to x_0}a^{f(x)} = a^l
  \end{equation*}
\end{potenzaLimiti}
\begin{potenzaLimiti1}
  Se $\lim\limits_{x\to x_0}f(x)=l>0$ e $a\in\mathbb{R}$
  \begin{equation*}
    \lim\limits_{x\to x_0}[f(x)]^a = l^a
  \end{equation*}
\end{potenzaLimiti1}
\begin{potenzaLimiti2}
  Se $\lim\limits_{x\to x_0}f(x)=l_1>0$ e $\lim\limits_{x\to x_0}g(x)=l_2$
  \begin{equation*}
    \lim\limits_{x\to x_0}[f(x)]^{g(x)} = l_1^{l_2}
  \end{equation*}
\end{potenzaLimiti2}

\subsubsection{Modulo}
\begin{moduloLimiti}
  Se $\lim\limits_{x\to x_0}=l$
  \begin{equation*}
    \lim\limits_{x\to x_0}\abs{f(x)}=\abs{l}
  \end{equation*}
\end{moduloLimiti}

\subsubsection{Logaritmo}
\begin{logLimiti}
  Se $\lim\limits_{x\to x_0}=l>0$ e $a\in\mathbb{R}_0^+\setminus\{1\}$
  \begin{equation*}
    \lim\limits_{x\to x_0}\log_af(x)=\log_al
  \end{equation*}
\end{logLimiti}

\subsection{Forme indeterminate}
Le forme indeterminate si ottengono quando si cerca di fare operazioni con limiti all'infinito. Le
forme indeterminate indicano che la sola conoscenza dei limiti delle due funzioni non determina la 
conoscenza del limite della loro operazione.\\
Quelle che vengono riquadrate di seguito sono le forme indeterminate nei vari casi.
\subsubsection{Somma}
\begin{center}
  \begin{tabular}{ccc}
    $\boldsymbol{\lim f(x)}$ & $\boldsymbol{\lim g(x)}$ & $\boldsymbol{\lim[f(x)+g(x)]}$\\\hline
    $l$ & $+\infty$ & $+\infty$\\
    $l$ & $-\infty$ & $-\infty$\\
    $\pm\infty$ & $\pm\infty$ & $\pm\infty$\\
    $\pm\infty$ & $\mp\infty$ & $\boxed{+\infty-\infty}$
  \end{tabular}
\end{center}

\subsubsection{Prodotto}
\begin{center}
  \begin{tabular}{ccc}
    $\boldsymbol{\lim f(x)}$ & $\boldsymbol{\lim g(x)}$ & $\boldsymbol{\lim[f(x)\cdot 
    g(x)]}$\\\hline
    $l\neq0$ & $\pm\infty$ & $\pm\infty$\\
    $\pm\infty$ & $\pm\infty$ & $\infty$\\
    $0$ & $\pm\infty$ & $\boxed{0\cdot\infty}$
  \end{tabular}
\end{center}

\subsubsection{Quoziente}
\begin{center}
  \begin{tabular}{ccc}
    $\boldsymbol{\lim f(x)}$ & $\boldsymbol{\lim g(x)}$ &
    $\boldsymbol{\lim\frac{f(x)}{g(x)}}$\\\hline
    $l$ & $\pm\infty$ & $0$\\
    $\pm\infty$ & $l\neq0$ & $\pm\infty$\\
    $\pm\infty$ & $\pm\infty$ & $\boxed{\frac{\pm\infty}{\pm\infty}}$\\
    $0$ & $0$ & $\boxed{\frac{0}{0}}$
  \end{tabular}
\end{center}

\subsubsection{Potenza}
\begin{center}
  \begin{tabular}{ccc}
    $\boldsymbol{\lim f(x)}$ & $\boldsymbol{\lim g(x)}$ &
    $\boldsymbol{\lim[f(x)]^{g(x)}}$\\\hline
    $l$ & $\pm\infty$ & $\pm\infty$\\
    $1$ & $\pm\infty$ & $\boxed{1^{\pm\infty}}$\\
    $+\infty$ & $0$ & $\boxed{+\infty^0}$\\
    $0$ & $0$ & $\boxed{0^0}$
  \end{tabular}
\end{center}

Per la risoluzione delle forme indeterminate, si utilizzino i limiti di una funzione razionale o 
limiti notevoli.

\subsection{Limite finito di una funzione razionale}
\begin{limiteFinitoFunzRaz}
  Quando $x$ tende a $x_0$, il limite di un polinomio coincide con il limite calcolato con 
  sostituzione.
\end{limiteFinitoFunzRaz}

Prendiamo ad esempio
\begin{equation*}
  \lim\limits_{x\to3^+}\frac{x^2-5x+6}{(x-3)^2}
\end{equation*}
Se provassimo a sostituire otterremmo
\begin{equation*}
  \lim\limits_{x\to3^+}\frac{x^2-5x+6}{(x-3)^2} = \frac{0}{0}
\end{equation*}
che è una forma indeterminata. In questa situazione si usa \hyperref[ruffini]{Ruffini} per ridurlo di
grado e ottenere
\begin{equation*}
  \lim\limits_{x\to3^+}\frac{x^2-5x+6}{(x-3)^2} = 
  \lim\limits_{x\to3^+}\frac{\cancel{(x-3)}(x-2)}{\cancel{(x-3)}(x-2)}
\end{equation*}
e per i teoremi dei limiti
\begin{equation*}
  \lim\limits_{x\to3^+}\frac{(x-2)}{(x-2)} = +\infty
\end{equation*}
In generale quindi
\begin{equation*}
  \lim\limits_{x\to x_0}\frac{p(x)}{q(x)} \overset{\left[\frac{0}{0}\right]}{=} =
  \lim\limits_{x\to x_0}\frac{p_1(x)}{q_1(x)} = \dotsb = \begin{cases}
    \text{Ruffini se }\frac{0}{0}\\
    l
  \end{cases}
\end{equation*}

\subsection{Limite all'infinito di una funzione razionale}
\hypertarget{teor:limiteInfinitoFunzRaz}{}
\begin{limiteInfinitoFunzRaz}
  Quando $x$ tende a $\pm\infty$, il limte di un polinomio coincide con il limite del suo monomio di
  grado più alto.
\end{limiteInfinitoFunzRaz}
Ad esempio consideriamo
\begin{equation*}
  \lim\limits_{x\to+\infty}(3x^3-5x^2+4x+1)
\end{equation*}
Poiché per $x\neq0$
\begin{equation*}
  3x^3-5x^2+4x+1=x^3\left(3-\frac{5}{x}+\frac{4}{x^2}+\frac{1}{x^3}\right)
\end{equation*}
si ha
\begin{equation*}
  \lim\limits_{x\to+\infty}\left(3-\frac{5}{x}+\frac{4}{x^2}+\frac{1}{x^3}\right) = 3
\end{equation*}
allora
\begin{equation*}
  \lim\limits_{x\to+\infty}(3x^3-5x^2+4x+1)=\lim\limits_{x\to+\infty}(3x^3) = +\infty
\end{equation*}
Se invece abbiamo una frazione, abbiamo
\begin{align*}
  &\lim\limits_{x\to\infty}\frac{a_nx^n+a_{n-1}x^{n-1}+\dotsb+a_0}{b_mx^m+b_{m-1}x^{m-1}+\dotsb+b_0}=
  \lim\limits_{x\to\infty}\frac{a_nx^n}{b_mx^m} = \\
  &\lim\limits_{x\to\infty}\left(\frac{a_n}{b_m}x^{n-m}\right) =
  \begin{dcases}
    \infty, &\text{se } n > m\\
    \frac{a_n}{b_n}, &\text{se } n = m\\
    0, &\text{se } n < m
  \end{dcases}
\end{align*}

\subsection{Limiti di funzioni irrazionali}
Creiamo questa sottosezione per la particolarità che i limiti con radicali possono avere. Prendiamo ad
esempio
\begin{equation*}
  \lim\limits_{x\to-\infty}\left(\sqrt{4x^2+1}-x\right)
\end{equation*}
Notiamo subito che se proviamo a sostituire, otteniamo la forma indeterminata
\begin{equation*}
  -\infty+\infty
\end{equation*}
Per risolvere questo tipo di limite, isoliamo il termine di grado massimo (quello che cresce più 
velocemente). Quindi
\begin{equation*}
  \lim\limits_{x\to+\infty}\left(\sqrt{4x^2+1}-x\right) = 
  \lim\limits_{x\to+\infty}\left(\sqrt{x^2\left(4+\frac{1}{x^2}\right)}-x\right)
\end{equation*}
Ora possiamo portare fuori $x^2$ dalla radice
\begin{equation*}
  \lim\limits_{x\to+\infty}\left(\sqrt{x^2\left(4+\frac{1}{x^2}\right)}-x\right) =
  \lim\limits_{x\to+\infty}\left(\abs{x}\sqrt{4+\frac{1}{x^2}}-x\right)
\end{equation*}
Ora possiamo sostituire $\abs{x} = x$ perché siamo in un intorno di $+\infty$. Questo perché sono 
numeri sicuramente $>0$ quindi il loro valore assoluto è esattamente lo stesso loro. (Se fossimo in un
$I(-\infty)$ sostituiremmo $\abs{x} = -x$). Proseguendo nella risoluzione
\begin{align*}
  &\lim\limits_{x\to+\infty}\left(\abs{x}\sqrt{4+\frac{1}{x^2}}-x\right)=
  \lim\limits_{x\to+\infty}\overbrace{x}^{\mathclap{\to+\infty}}
  \overbrace{\left(\sqrt{4+\frac{1}{x^2}}-x\right)}^{\mathclap{\to\sqrt{4+0}-1\to2-1\to1}}=\\
  &+\infty\cdot1 = \boxed{+\infty}
\end{align*}
In generale, quindi, si deve sempre isolare il termine che cresce più rapidamente utilizzando a proprio
favore l'operatore $\lim$.

\subsection{Limiti notevoli}
Ci sono dei limiti particolari che è estremamente comodo conoscere a memoria. Principalmente sono 2
\begin{equation*}
  \lim\limits_{x\to0}\frac{\sin x}{x}=1\qquad\lim\limits_{x\to+\infty}\left(1+\frac{1}{x}\right)^x=e
\end{equation*}
Da questi due se ne possono ricavare altri 6. Dal primo
\begin{equation*}
  \lim\limits_{x\to0}\frac{\tan x}{x}=1\quad\lim\limits_{x\to0}\frac{1-\cos x}{x^2}=\frac{1}{2}\quad
  \lim\limits_{x\to0}\frac{1-\cos x}{x}=0
\end{equation*}
Dal secondo
\begin{equation*}
  \lim\limits_{x\to0}(1+x)^{\frac{1}{x}}=e\,\lim\limits_{x\to0}\frac{a^x-1}{x}=\ln a\,
  \lim\limits_{x\to0}\frac{\log_a(x+1)}{x}=\log_a e
\end{equation*}
Infine ne esiste un ultimo che è molto simile al primo ma non identico
\begin{equation*}
  \lim_{x \to \infty} \frac{\sin x}{x} = 0 
\end{equation*}


\subsection{Consigli nella risoluzione di limiti deducibili}
Prendiamo ad esempio
\begin{equation*}
  \lim\limits_{x\to4^-}\frac{1}{\log_2 x -2}
\end{equation*}
Per trovare questo limite, sostituiamo $x = 4$ nella funzione. Otteniamo
\begin{equation*}
  \lim\limits_{x\to4^-}\frac{1}{\log_2 2} \to \lim\limits_{x\to4^-}\frac{1}{1} = 1
\end{equation*}
Abbiamo molto semplicemente trovato il limite sostituendo. Spesso però bisogna anche stare attenti 
da che parte $x$ tende.\\[\baselineskip]
Prendiamo un altra funzione
\begin{equation*}
  \lim\limits_{x\to2^+}\sqrt{\log_2\frac{x+2}{x-2}}
\end{equation*}
Questa può far paura ma andando con calma, scopriamo che non è affatto difficile. Dobbiamo un po'
lavorare come con i domini: dall'interno all'esterno. Con valori numerici, sostiuiamo ad $x$ il valore
di $x_0$
\begin{equation*}
  \lim\limits_{x\to2^+}x+2 \to 2^++2 \to 4^+
\end{equation*}
(Il segno qui non è obbligatorio da mantenere in quanto stiamo lavorando su $4$, se invece stessimo
usando $0$, è determinante in quanto può cambiare il risultato).
\begin{equation*}
  \lim\limits_{x\to2^+}x-2 \to 2^+-2 \to 0^+
\end{equation*}
(Il segno invece qui è indispensabile, ora capiremo perché)
\begin{equation*}
  \lim\limits_{x\to2^+}\frac{x+2}{x-2}\to\lim\limits_{x\to2^+}\frac{4^+}{0^+}\to+\infty
\end{equation*}
Se avessimo avuto $0^-$ non ci saremmo più avvicinati da destra, ma da sinistra e quindi avremmo 
ottenuto $-\infty$ in quanto il grafico di $\frac{1}{x}$ per $x\to 0^+$ tende all'infinito positivo,
per $x\to0^-$ a quello negativo. Proseguendo
\begin{equation*}
  \lim\limits_{x\to2^+}\log_2 +\infty \to +\infty
\end{equation*}
Questo lo si può capire molto facilmente dal grafico (si veda sopra). Ecco perché conoscere i grafici
generali delle funzioni più comuni è molto comodo.
\begin{equation*}
  \lim\limits_{x\to2^+}\sqrt{+\infty}\to+\infty
\end{equation*}
Quindi, con questo
\begin{equation*}
  \lim\limits_{x\to2^+}\sqrt{\log_2\frac{x+2}{x-2}} = +\infty
\end{equation*}
In definitiva quindi il consiglio è di andare con molta calma e ricordarsi le possibilità che la
funzione $\lim$ offre.

\subsection{Asintoti}
Un asintoto è quella retta a cui la funzione tende sempre di più. Sia $f(x)$ la funzione in 
questione, allora $G(f)$ è il suo grafico. Se $r$ è la retta che stiamo cercando, $P\in G(f)$,
$PH$ è la distanza punto-retta. Si ha quindi
\begin{equation*}
  \lim_{x\to\infty}\overline{PH}=0 
\end{equation*}
Ci sono 3 tipi di asintoti: \emph{verticali},  \emph{orizzontali} e  \emph{obliqui}.

\subsubsection{Verticali}
\begin{equation*}
  x=x_0\quad \lim_{x\to x_0^{\pm}}f(x)=\pm\infty
\end{equation*}
Una funzione può avere nessuno, 1 o infiniti asintoti verticali.

\subsubsection{Orizzontali}
\begin{equation*}
  y=l\quad \lim_{x\to\pm\infty}f(x)=l
\end{equation*}
Perché ci siano asintoti orizzontali è necessario che negli intorni di $\pm\infty$ il dominio sia
illimitato.

\subsubsection{Obliqui}
\begin{equation*}
  y=mx+q\quad m =\lim_{x\to\pm\infty}\frac{f(x)}{x}\quad 
  q=\lim_{x\to\pm\infty}\left( f(x)-mx \right) 
\end{equation*}
Per trovare gli asintoti obliqui prima si trovi $m$ e poi la si inserisca per trovare $q$.

\subsubsection{Generalizzazione}
In generale, una funzione $\alpha(x)$ si può definire asintoto di una funzione $\beta(x)$ se
\begin{equation*}
  \lim_{x\to\infty}\left( \alpha(x) - \beta(x) \right) = 0
\end{equation*}
Quindi se la differenza tra le ordinate della funzione asintotica ($\alpha$) e della funzione 
cercata ($\beta$) è $0$ per un intorno di $\infty$.\\ [\baselineskip]
Inoltre si può anche dire che un asintoto obliquo sia il quoziente tra un polinomio $f(x)$ e un 
altro $g(x)$. Da un quoziente di polinomi otteniamo un resto e un quoziente. Ovvero
\begin{equation*}
  y = \frac{f(x)}{g(x)}\quad f(x) = g(x)\cdot q(x)+r(x)
\end{equation*}
sistemando la seconda equazione otteniamo
\begin{equation*}
  \frac{f(x)}{g(x)}-q(x) = \frac{r}{g(x)}
\end{equation*}
questo ci viene molto utile specialmente dato che
\begin{equation*}
  \lim_{x\to\infty} \left[ \frac{f(x)}{g(x)}-q(x) \right] = \lim_{x\to\infty} \frac{r}{g(x)} = 0
\end{equation*}
In definitiva si segua
\begin{itemize}
  \item Eseguire la divisione fra polinomi
  \item Ottenere il resto e il quoziente
\end{itemize}
Il quoziente che otterremo sarà la funzione asintotica quella data. Per trovare see l'asintoto
è superiore o inferiore alla funzione, si trovi il segno di $\frac{r}{g(x)}$. Se è $>0$ anche il
grafico della funzione è sopra l'asintoto, sotto altrimenti. \\ [\baselineskip]
Da questo si ricava una cosa molto interessante ovvero che se $f(x)$ è di grado 3 e $g(x)$ è 
di primo grado, otteniamo che il quoztiente viene di grado secondo, quindi l'asintoto non è
una retta, ma è una parabola!
