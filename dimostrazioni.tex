%!TEX ROOT=formularioMatematica.tex
\section{Dimostrazioni}
Qui verranno inserite alcune dimostrazioni di teoremi o formule che vengono usate nel formulario.

\begin{proof}[\protect\hyperlink{teor:tfa-ext}{Teorema fondamentale dell'Algebra esteso}]
	Il polinomio $P(x)$ in virt� del teorema fondamentale dell'Algebra, ha in $\mathbb{C}$ almeno uno 
	zero. Indicato con $\alpha_1$ tale zero, risulta:
	\begin{equation*}
	P(x) = (x-\alpha_1)P_1(x)
	\end{equation*}
	essendo il quoziente $P_1(x)$ un polinomio, a coefficienti in $\mathbb{C}$, di grado $(n-1)$.\\
	Se $n-1>0$, allora, per il teorema fondamentale dell'Algebra, anche il polinomio $P_1(x)$ ha in
	$\mathbb{C}$ almeno uno zero. Indicando tale zero con $\alpha_2$ avremo:
	\begin{equation*}
	P_1(x)=(x-\alpha_2)P_2(x)
	\end{equation*}
	essendo il quoziente $P_2(x)$ un polinomio, a coefficienti in $\mathbb{C}$, di grado $(n-2)$.\\
	Risulta quindi:
	\begin{align*}
	P(x)&=\underbrace{(x-\alpha_1)(x-\alpha_2)\dotsm(x-\alpha_n)P_n(x)}_{n\text{ fattori}} = \\
	&(x-\alpha_1)(x-\alpha_2)\dotsm(x-\alpha_n)c
	\end{align*}
	essendo l'ultimo termine di grado zero pari ad una costante $c$.\\
	Poich� la costante $c$ � il coefficiente del termine di grado massimo $x^n$, ne segue che $c=a_n$
	da cui
	\begin{equation*}
	P(x) = a_n(x-\alpha_1)(x-\alpha_2)\dotsm(x-\alpha_n)
	\end{equation*}
\end{proof}

\begin{proof}
	[\protect\hyperlink{teor:limiteInfinitoFunzRaz}{Limite di una funzione razionale}]
	Se
	\begin{equation*}
	P(x)=a_nx^n+a_{n-1}x^{n-1}+\dotsb+a_0
	\end{equation*}
	� un polinomio di grado $n>0$, si pu� scrivere per $x\neq0$
	\begin{equation*}
	P(x) = x^n\left(a_n+\frac{a_{n-1}}{x}+\dotsb+\frac{a_0}{x^n}\right)
	\end{equation*}
	e quindi, poich� $\lim\limits_{x\to+\infty}\frac{1}{x^n}=0,\,\forall n\in\mathbb{N}_0$, risulta
	\begin{equation*}
	\lim\limits_{x\to\infty}x^n\left(a_n+\frac{a_{n-1}}{x}+\dotsb+\frac{a_0}{x^n}\right) = a_n
	\end{equation*}
	Si ha
	\begin{align*}
	\lim\limits_{x\to\infty}P(x)&=\lim\limits_{x\to\infty}=
	\lim\limits_{x\to\infty}\left(a_nx^n+a_{n-1}x^{n-1}+\dotsb+a_0\right)=\\
	&\lim\limits_{x\to\infty}\left(a_nx^n\right)
	\end{align*}
\end{proof}

\begin{proof}[\protect\hyperlink{teor:uniLim}{Unicit� del limite}]
	Supponiamo per assurdo che la funzione $f$ per $x\to x_0$ ammetta due limiti distinti $l_1$ e 
	$l_2$, cio� che si abbia
	\begin{equation*}
	\lim\limits_{x\to x_0}f(x)=l_1\quad\lim\limits_{x\to x_0}f(x)=l_2
	\end{equation*}
	In base alla definizione di limite, preso comunque un numero $\varepsilon>0$, � possibile 
	determinare due numeri positivi $\delta_\varepsilon'$ e $\delta_\varepsilon''$ tali che, per ogni
	$x\in\mathscr{D}_f$, verificante la condizione
	\begin{align*}
	0&<\abs{x-x_0}<\delta_\varepsilon' \quad\text{risulti}\quad \abs{f(x)-l_1}<\varepsilon\\
	0&<\abs{x-x_0}<\delta_\varepsilon'' \quad\text{risulti}\quad\abs{f(x)-l_2}<\varepsilon\\
	\end{align*}
	Ora, sia $\delta_\varepsilon$ il minore tra i due numeri $\delta_\varepsilon',\delta_\varepsilon''$
	per
	\begin{equation*}
	0<\abs{x-x_0}<\delta_\varepsilon
	\end{equation*}
	risulteranno verificate entrambe le disequazioni precedenti e potremo scrivere
	\begin{equation*}
	\abs{l_1-l_2}=\abs{l_1-f(x)+f(x)-l_2}\leq\abs{f(x)-l_1}+\abs{f(x)-l_2}<\varepsilon+\varepsilon=
	\varepsilon2
	\end{equation*}
	Data l'arbitrariet� di $\varepsilon$, la condizione $\abs{l_1-l_2}<2\varepsilon$ implica che sia
	$\abs{l_1-l_2}=0$ cio� $l_1=l_2$.
\end{proof}

\begin{proof}[\protect\hyperlink{teor:confLim}{Teorema del confronto}]
	In base alla definizione di limite, preso comunque un numero $\varepsilon>0$, � possibile 
	determinare due numeri positivi $\delta_\varepsilon'$ e $\delta_\varepsilon''$ tali che, per ogni
	$x\in\mathscr{D}_f$, verificante la condizione
	\begin{alignat*}{2}
	0&<\abs{x-x_0}<\delta_\varepsilon' &\quad\text{risulti}\quad \abs{f(x)-l_1}<\varepsilon\\
	0&<\abs{x-x_0}<\delta_\varepsilon'' &\text{risulti} \abs{f(x)-l_2}<\varepsilon\\
	\end{alignat*}
	Ora, sia $\delta_\varepsilon$ il minore tra i due numeri $\delta_\varepsilon',\delta_\varepsilon''$
	per
	\begin{equation*}
	0<\abs{x-x_0}<\delta_\varepsilon
	\end{equation*}
	saranno verificate entrambe le disequazioni precedenti quindi
	\begin{equation*}
	l-\varepsilon<f(x)\leq g(x)\leq h(x)<l+\varepsilon
	\end{equation*}
	cio�
	\begin{equation*}
	\abs{g(x)-l}<\varepsilon
	\end{equation*}
\end{proof}

\begin{proof}[\protect\hyperlink{teor:segno}{Teorema della permanenza del segno}]
	Dimostriamo innanzitutto la prima parte del teorema.\\
	Sia $\varepsilon=\frac{\abs{l}}{2}$; per la definizione di limite � possibile determinare in 
	corrispondenza di tale $\varepsilon$, un numero $\delta_\varepsilon>0$ tale che se 
	$x\in\mathscr{D}_f$
	\begin{equation*}
	0<\abs{x-x_0}<\delta_\varepsilon\quad\text{implichi}\quad
	l-\frac{\abs{l}}{2}<f(x)<l+\frac{\abs{l}}{2}
	\end{equation*}
	Ne consegue la tesi non appena si osservi che
	\begin{equation*}
	\text{se}\, l<0\quad l+\frac{\abs{l}}{2}<0\quad\text{quindi}\quad f(x)<0
	\end{equation*}
	\begin{equation*}
	\text{se}\, l>0\quad l-\frac{\abs{l}}{2}>0\quad\text{quindi}\quad f(x)>0
	\end{equation*}
	Dimostriamo ora la seconda parte.\\
	Sia per esempio $f(x)>0$. Dalla definizione di limite � possibile determinare in 
	corrispondenza di tale $\varepsilon$, un numero $\delta_\varepsilon>0$ tale che se 
	$x\in\mathscr{D}_f$
	\begin{equation*}
	0<\abs{x-x_0}<\delta_\varepsilon\quad\text{implichi}\quad
	l-\varepsilon<f(x)<l+\varepsilon
	\end{equation*}
	Supponiamo ora per assurdo che sia $l<0$, scegliendo $\varepsilon=-\frac{l}{2}>0$; si avrebbe
	\begin{equation*}
	f(x)<\frac{l}{2}<0
	\end{equation*}
	contro l'ipotesi che sia $f(x)>0$. Sar� dunque $l\geq0$
\end{proof}

\begin{proof}[\protect\hyperlink{teor:sommaLimiti}{Limite di una somma}]
	Si ha intanto
	\begin{equation*}
	\abs{[f(x)+g(x)]}-(l_1+l_2)\leq\abs{f(x)-l_1}+\abs{g(x)-l)2}
	\end{equation*}
	e quindi, preso $\varepsilon>0$, se si vuol provare che il primo membro � pi� piccolo di 
	$\varepsilon$, basta verificare che ciascuno dei due addendi a secondo membro � pi� piccolo di
	$\frac{\varepsilon}{2}$.\\
	Ma questo � evidente per le definizioni stesse di limiti. Il primo addendo sar� minore di 
	$\frac{\varepsilon}{2}$ se
	\begin{equation*}
	0<\abs{x-x_0}<\delta_\varepsilon'
	\end{equation*}
	e il secondo se
	\begin{equation*}
	0<\abs{x-x_0}<\delta_\varepsilon''
	\end{equation*}
	ove i due numeri $\delta_\varepsilon'$ e $\delta_\varepsilon''$ possono essere diversi in quanto 
	si riferiscono a funzioni diverse.\\
	Detto allora $\delta_\varepsilon$ il minore dei due, scegliendo $x$ tale che
	\begin{equation*}
	0<\abs{x-x_0}<\delta_\varepsilon
	\end{equation*}
	si soddisfano entrambe le condizioni; quindi per valori di $x$ cos� scelti si avr�
	\begin{equation*}
	\abs{f(x)-l_1}<\frac{\varepsilon}{2}\quad\abs{g(x)-l_2}<\frac{\varepsilon}{2}
	\end{equation*}
	e di conseguenza
	\begin{equation*}
	\abs{[f(x)+g(x)]}-(l_1+l_2)<\varepsilon
	\end{equation*}
\end{proof}

\begin{proof}[\protect\hyperlink{teor:prodottoLimiti}{Limite di un prodotto}]
	Si ha
	\begin{align*}
	&\abs{f(x)\cdot g(x)-l_1\cdot l_2} = \\
	&\abs{f(x)\cdot g(x)+l_1\cdot g(x)-l_1\cdot g(x)-l_1\cdot l_2} =\\
	&\abs{g(x)\cdot(f(x)-l_1)+l_1\cdot(g(x)-l_2)}\leq\\
	&\abs{g(x)}\cdot\abs{f(x)-l_1}+\abs{l_1}\cdot
	\abs{g(x)-l_2}
	\end{align*}
	Fissato allora $\varepsilon'$ in modo che sia $0<\varepsilon'<1$, esiste in corrispondenza di esso 
	un numero positivo $\delta_{\varepsilon'}$ tale che, $\forall x\in I$ verificante la condizione
	$0<\abs{x-x_0}<\delta_{\varepsilon'}$, risulti
	\begin{equation*}
	\abs{f(x)-1}<\varepsilon'\quad\abs{g(x)-l_2}<\varepsilon'\quad\abs{g(x)}<\abs{l_2}+\varepsilon'
	\end{equation*}
	Si ricava quindi
	\begin{equation*}
	\abs{f(x)\cdot g(x)-l_1\cdot l_2}<(\abs{l_2}+\varepsilon')\varepsilon'+\abs{l_2}\varepsilon'<
	(\abs{l_2}+\abs{l_1}+1)\varepsilon'
	\end{equation*}
	poich� $\varepsilon'^2<\varepsilon'$, essendo $0<\varepsilon'<1$, se scegliamo $\varepsilon'$ non
	solo positivo e minore di $1$ ma anche minore di
	\begin{equation*}
	\frac{\varepsilon}{\abs{l_1}+\abs{l_2}+1}
	\end{equation*}
	si ottiene, per $x$ appartenente ad un opportuno intorno di $x_0$
	\begin{equation*}
	\abs{f(x)\cdot g(x)-l_1\cdot l_2}<\varepsilon
	\end{equation*}
\end{proof}
