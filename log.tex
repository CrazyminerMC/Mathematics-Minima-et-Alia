%!TEX ROOT=formularioMatematica.tex

\section{Logaritmi}\label{sec:logaritmi}
Il logaritmo � la seconda funzione inversa della potenza, essendo la prima la radice.\\
Presa un'equazione del tipo
\begin{equation*}
a^x=b
\end{equation*}
le soluzioni di $x$ si esprimono come
\begin{equation*}
x = \log_a b
\end{equation*}
quindi si ha anche che 
\begin{equation*}
a^{\log_a b} = b
\end{equation*}
Si legge ``\textit{logaritmo in base $a$ di $b$}''. Perch� un logaritmo esista � necessario che
$a>0\land a\neq1\land b > 0$.\\
Quando si vede scritto $\log$ si intende $\log_{10}$, quando invece � presente $\ln$ si intende
$\log_e$.\\
Per gli esercizi si vada \hyperref[ex:logaritmi]{qui}.

\subsection{Teoremi sui logaritmi}
\subsubsection{Logaritmo del prodotto}
\begin{equation*}
\log_a (b_1\cdot b_2) = \log_a b_1 + \log_a b_2
\end{equation*}

\subsubsection{Logaritmo del quoziente}
\begin{equation*}
\log_a\frac{b_1}{b_2} = \log_a b_1 - \log_a b_2
\end{equation*}

\subsubsection{Logaritmo di una potenza}
\begin{equation*}
\log_a b^k = k\log_a b
\end{equation*}

\subsubsection{Cambiamento di base}
\begin{equation*}
\log_a b = \frac{\log_c b}{\log_c a}
\end{equation*}
Da questa particolare formula si nota anche che
\begin{equation*}
\log_{\frac{1}{a}} b = -\log_a b
\end{equation*}

\subsection{Grafici dei logaritmi}
I logaritmi hanno due grafici dipendentemente al valore della base
\subsubsection{$\log_a x$ con $a > 0$}
\begin{center}
	\begin{tikzpicture}
		\tkzInit[xmin=-1,ymin=-2.5,xmax=5,ymax=3]
		\tkzGrid
		\tkzAxeXY
		\draw[red, thick, domain=0.1:5, samples=500] plot({\x}, {ln(\x)});
	\end{tikzpicture}
\end{center}

\subsubsection{$\log_a x$ con $0 < a < 1$}
\begin{center}
	\begin{tikzpicture}
	\tkzInit[xmin=-1,ymin=-2.5,xmax=5,ymax=3]
	\tkzGrid
	\tkzAxeXY
	\draw[red, thick, domain=0.1:5, samples=500] plot({\x}, {ln(\x)/ln(0.5)});
	\end{tikzpicture}
\end{center}
Essendo i logaritmi molto correlati alle funzioni esponenziali, riporto di seguito i loro grafici

\subsubsection{$a^x$ con $a > 1$}
\begin{center}
	\begin{tikzpicture}
	\tkzInit[xmin=-3,ymin=-1,xmax=2,ymax=4.5]
	\tkzGrid
	\tkzAxeXY
	\draw[red, thick, domain=-3:1.5, samples=500] plot({\x}, {exp(\x)});
	\end{tikzpicture}
\end{center}

\subsubsection{$a^x$ con $0 < a < 1$}
\begin{center}
	\begin{tikzpicture}
	\tkzInit[xmin=-2,ymin=-1,xmax=3,ymax=4]
	\tkzGrid
	\tkzAxeXY
	\draw[red, thick, domain=-2:3, samples=500] plot({\x}, {0.5^(\x)});
	\end{tikzpicture}
\end{center}