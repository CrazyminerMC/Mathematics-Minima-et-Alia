%!TEX ROOT=formularioMatematica.tex

\section{Distribuzioni}
Il concetto di fondo che sta alla base delle distribuzioni sono le \textbf{variabili casuali}. Ci 
sono due tipi di variabili casuali: discrete e continue e con esse due tipi di distribuzioni. Le
variabili si indicano con una $X$, maiuscola.

\subsection{Distribuizioni discrete}
Una variabile discreta è una variabile che può assumere un numero finito o numerabile di valori. Ad
ogni variabile si associa una certa probabilità. Spesso si impostano le variabili sottoforma di 
tabella in modo da evidenziare ciascuna probabilità
\begin{center}
  \begin{tabular}{c c c c c}
    $X$: & $X_1$ & $X_2$ & $\cdots$ & $X_n$\\\midrule
    $p$: & $p_1$ & $p_2$ & $\cdots$ & $p_n$
  \end{tabular}
\end{center}
È ovvio che $\sum p_i = 1$.\\ [\baselineskip]
Si definiscono due funzioni: \textbf{funzione di distribuzione di probabilità} e di i
\textbf{ripartizione di probabilità}, rispettivamente
\begin{equation*}
  f(x_i) = P(X=x_i)
\end{equation*}
e
\begin{equation*}
  F(x) = P(X\leq x)\quad\forall x\in\mathbb{R}
\end{equation*}
È da notare che per il modo in cui sono definite, si hanno i seguenti due grafici
\begin{center}
  \begin{tikzpicture}[scale=0.75]
    \begin{axis}[xmin=0,ymin=0,xmax=4,ymax=4,ticks=none,ylabel=$f(x)$]
      \draw[thick] (1,0) -- (1,1);
      \draw[thick] (2,0) -- (2,3);
      \draw[thick] (3,0) -- (3,2);
    \end{axis}
  \end{tikzpicture}
\end{center}
e così via per tutti gli $x_i$.
\begin{center}
  \begin{tikzpicture}[scale=0.75]
    \begin{axis}[xmin=0,ymin=0,xmax=4,ymax=4,ticks=none,ylabel=$F(x)$]
      \draw[thick] (1,1) circle (0.05);
      \draw[thick] (1,1) -- (2,1); 
      \draw[thick] (2,1) circle (0.05);
      
      \draw[thick] (2,2) circle (0.05);
      \draw[thick] (2,2) -- (3,2);
      \draw[thick] (3,2) circle (0.05);

      \draw[thick] (3,3) circle (0.05);
      \draw[thick] (3,3) -- (4,3);
      \draw[thick] (4,3) circle (0.05);
    \end{axis}
  \end{tikzpicture}
\end{center}
Da questi grafici si evincono alcune cose:
\begin{enumerate}
  \item $F(x)=P(X\leq x)=\sum\limits^{n}_{x_i<x} f(x_i)$
  \item $f(x_i)\geq0\quad\forall x_i$
  \item $\sum f(x_i)=1$
  \item $F(x)$ è monotona non decrescente
  \item $0\leq F(x_i)\leq1$
\end{enumerate}
Si definiscono anche altre due funzioni estremamente usate: il valore medio (o indice di media o
valore aspettato) che è definita come
\begin{equation*}
  E(X) = \sum\limits^{n}_{i=1} x_if(x_i)
\end{equation*}
e la varianza
\begin{equation*}
  \sigma^2(X) = \sum\limits^{n}_{i=1} [x_i-E(X_i)]^2f(x_i) = E(X^2)-E^2(X)
\end{equation*}

\subsubsection{Distribuzione binomiale}
La più comune distribuzione discreta è forse la binomiale. Il requisito è che l'esperimento, 
ripetuto in medesime condizioni, ottenga solo 2 risultati: \textbf{successo} e \textbf{insuccesso}.
La probabilità del successo si definisce $p$, quella dell'insuccesso $q=1-p$.\\
Spesso si definisce anche $p_{n,k}$ ovvero la probabilità che su $n$ esperimenti, $k$ siano 
successi. Andando quindi a tabulare $X\sim B(n,p)$, ovvero la variabile $X$ distribuita secondo la
binomiale di $n$ esperimenti con probabilità $p$, si ottiene
\begin{center}
  \begin{tabular}{c c c c}
    $X$: & $0$ & $k$ & $n$\\\midrule
    $p$: & $q^n$ & $\binom{n}{k}p^kq^{n-k}$ & $p^n$
  \end{tabular}
\end{center}
Questo perché ovviamente se ci sono $0$ successi, la probabilità è quella di tutti insuccessi. Se 
ci sono $n$ successi invece la probabilità è quella di tutti successi. Nel mezzo ritroviamo il
coefficiente binomiale in quanto sono combinazioni (i modi in cui successi ed insuccessi si possono
distribuire) di probabilità.\\ [\baselineskip]
Quindi abbiamo che
\begin{equation*}
  p_{n,k} = \binom{n}{k} p^kq^{n-k}
\end{equation*}
Inoltre si trova anche il valore medio
\begin{equation*}
  E(X) = np
\end{equation*}
e la varianza
\begin{equation*}
  \sigma^2(X) = npq
\end{equation*}

\subsection{Distribuzioni continue}
Le distribuzioni continue sono caratterizzate da variabili casuali le cui possibibilità sono molto
numerose o non numerabili. Infatti si dice che $X\in[a,b]$ e quindi può assumere tutti i valori.
Andando a disegnare per istogrammi una possibile distribuzione normale, si ottiene
\begin{center}
  \begin{tikzpicture}
    \begin{axis}[xmin=0,ymin=0,xmax=8,ymax=1,ticks=none,ylabel=$f$,axis equal image]
      \addplot[domain=0:4,smooth,samples=500,integral=0:8,integral segments=20,blue,thick] 
        {exp(-((x-2)^2)/(2*x))};
    \end{axis}
  \end{tikzpicture}
\end{center}
Qui rappresentata è la frequenza relativa, ovvero $\sum f_i = 1$. Quella assoluta assume un grafico
molto simile
\begin{center}
  \begin{tikzpicture}
    \begin{axis}[xmin=0,ymin=0,xmax=8,ymax=4,ticks=none,ylabel=$f$]
      \addplot[domain=0:4,smooth,samples=500,integral=0:8,integral segments=20,blue,thick] 
        {4*exp(-((x-2)^2)/(2*x))};
    \end{axis}
  \end{tikzpicture}
\end{center}
infatti si ha che $\sum f_i = n$. Conoscendo la $f(x)$ che descrive un andamento come quello che
si vede dagli istogrammi, si può trovare la probabilità che avvenga un evento all'interno di un
intervallo. Questa probabilità è infatti nient'altro che l'area tra i due punti
\begin{equation*}
  p(x_1<X<x_2) = \int\limits_{x_1}^{x_2} f(x)\dif x
\end{equation*}
Da questo si evince una cosa molto importante: $P(X=x)=0$. Infatti la probabilità che tra infinite
possibilità avvenga un determinato evento è $0$.\\ [\baselineskip]
La funzione di distribuzione di probabilità è definita molto semplicemente
\begin{equation*}
  f(x)=\int\limits_{-\infty}^{+\infty}f(x)\dif x 
\end{equation*}
Questo anche perché $f(x)>0$ per ogni $x$.\\
La funzione di ripartizione invece
\begin{equation*}
  F(X\leq x) = \int\limits_{-\infty}^{x} f(x)\dif x
\end{equation*}
È da notare che in ambito continuo scrivere $F(X<x)$ o $F(X\leq x)$ è assolutamente indifferente 
per la caratteristica vista sopra.\\
Si noti anche che $P(x_1\leq X\leq x_2)=F(x_2)-F(x_1)$.

\subsection{Tabella riassuntiva delle formule di distribuzioni discrete e continue}
\begin{center}
  \begin{tabular}{c|c|c}
    & Discreta & Continua\\
    $E(X)$ & $\sum f(x_i)x$ & $\int\limits_{-\infty}^{+\infty} f(x)\dif x$\\
    $\sigma^2(X)$ & $\sum[x_i-E(X_i)]^2f(x_i)$ & $\int\limits_{-\infty}^{x} [x-E(X)]^2f(x)\dif x$
  \end{tabular}
\end{center}

\subsection{Distribuzione Gaussiana (normale)}
\begin{center}
  \begin{tikzpicture}
    \begin{axis}[xmin=-3,ymin=0,xmax=9,ymax=5,axis equal,ticks=none]
      \addplot[domain=-3:9,thick,blue,smooth,samples=500] {6*exp((-(x-3)^2)/7)};
      \draw[thick,dashed] (3,0) -- (3,6) node[pos=0,below]{$\mu$};
      \draw[thick,dashed] (1.0664,0) -- (1.0664,3.5171)node[pos=0,below]{$\sigma$};
      \draw[thick,dashed] (4.9464,0) -- (4.9464,3.5171)node[pos=0,below]{$\sigma$};
    \end{axis}
  \end{tikzpicture}
\end{center}
Qui rappresentata è la funzione gaussiana, la distribuzione normale, una tra le più comuni. È 
definita tramite due parametri: $\mu=E(X)$ e $\sigma=\sqrt{\sigma^2(x)}$. La funzione quindi è
\begin{equation*}
  f(x) = \frac{1}{\sigma\sqrt{2\pi}}e^{-\frac{(x-\mu)^2}{2\sigma^2}}
\end{equation*}
I due punti di flesso si hanno a $x_F = \mu\pm\sigma$.\\
La peculiarità di questa funzione è che non ha primite note o esprimibili tramite funzioni 
elementari. Questo significa che per essere utilizzata si deve usufruire di tabelle. Tutte le
tabelle fanno riferimento alla gaussiana standardizzata o normale che è espressa tramite $N(0,1)$
ovvero hanno $\mu=0$ e $\sigma=1$ in modo da rendere la funzione nella forma
\begin{equation*}
  f(x) = \frac{1}{\sqrt{\pi}} e^{-\frac{x^2}{2}}
\end{equation*}
Tutte le gaussiane si possono ricondurre a questa standardizzata facendo la sostituzione
\begin{equation*}
  Z = \frac{x-\mu}{\sigma}
\end{equation*}
La caratteristica è che anche dopo questa trasformazione la funzione mantiene la stessa area. 
Quindi ha che se $X\sim N(\mu,\sigma)$ allora $Z\sim N(0,1)$ e quindi $P(x_1\leq X\leq x_2) =
P(z_1\leq Z\leq z_2)$.\\
È da notare un'ultima cosa
\begin{equation*}
  E(aX+b) = aE(X)+b
\end{equation*}
ed
\begin{equation*}
  \sigma^2(aX+b) = a^2\sigma^2(X)
\end{equation*}

\subsubsection{Uso delle tabelle}
Le tabelle offrono un'approssimazione dell'area. Si ha quindi che nella prima colonna sono
identificata le prime due cifre della $X$ d'interesse (nelle varie righe). Nelle successive invece
si trovano le cifre da $0$ e $9$ per completare il numero. In corrispondenza dell'incrocio tra
le informazioni, si trova l'approssimazione dell'area.
