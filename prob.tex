%!TEX ROOT=formularioMatematica.tex

\section{Probabilità}\label{sec:prob}
La probabilità è una funzione $p(U)$ che ritorna un valore compreso tra $0$ e $1$ che definisce la 
probilità di un \emph{evento}.
\begin{equation*}
p:\,p(U) = \frac{\text{Casi favorevoli}}{\text{Casi possibili}}\mapsto {[{0},{1}]}
\end{equation*}

\subsection{Evento ed insieme universo}
Per un qualsiasi caso di studio esiste un insieme \emph{Universo} definito $\mathbb{U}$ che contiene
tutte le possibili uscite dell'oservazione. Ciascuna di queste uscite è definito \emph{evento}.
Quindi
\begin{equation*}
\mathbb{E} \subseteq \mathbb{U}
\end{equation*}
e detto in altri termini, un evento è un insieme di possibilità. Ad esempio
\begin{equation*}
\mathbb{E} = \{2,4,5\}
\end{equation*}
può essere un evento nel lancio di un dado.\\
$p(\mathbb{U}) = 1$ per qualsiasi tipo di osservazione. Quindi la probabilità che \textbf{non} avvenga
un evento è $1-p(\mathbb{E})$

\subsubsection{Eventi incompatibili}
Due eventi si dicono incompatibili quando
\begin{equation*}
\mathbb{E}_1 \cap \mathbb{E}_2 = \emptyset
\end{equation*}

\subsubsection{Eventi indipendenti}
Due eventi si dicono indipendenti quando
\begin{equation*}
p\left(\mathbb{E}_1\mid\mathbb{E}_2\right) = p(\mathbb{E}_1)
\end{equation*}

\subsection{Probabilità di eventi incompatibili}
\begin{equation*}
p(\mathbb{E}_1\cup\mathbb{E}_2) = p(\mathbb{E}_1) + p(\mathbb{E}_2)
\end{equation*}

\subsection{Probabilità di eventi compatibili}
\begin{equation*}
p(\mathbb{E}_1\cup\mathbb{E}_2) = p(\mathbb{E}_1)+p(\mathbb{E}_2)-p(\mathbb{E}_1\cap\mathbb{E}_2)
\end{equation*}
Si estenda questa formula in modo che si tolgano tutte le intersezioni fra eventi per non ripetere 
risultati.

\subsection{Probabilità condizionata}
La probabilità condizionata indica la probabilità che si verifichi l'evento $\mathbb{E}_1$ 
verificatosi $\mathbb{E}_2$.
\begin{equation*}
p\left(\mathbb{E}_1\mid\mathbb{E}_2\right) = 
\frac{p\left(\mathbb{E}_1\cap\mathbb{E}_2\right)}{p(\mathbb{E}_2)}
\end{equation*}

\subsection{Probabilità composta}
Indica la probabilità che si verifichi un evento intersezione di altri due.
\begin{equation*}
p\left(\mathbb{E}_1\cap\mathbb{E}_2\right) = p(\mathbb{E}_1)\cdot
p\left(\mathbb{E}_1\mid\mathbb{E}_2\right)
\end{equation*}

Però se sono eventi indipendenti si semplifica in
\begin{equation*}
p\left(\mathbb{E}_1\cap\mathbb{E}_2\right) = p(\mathbb{E}_1)\cdot p(\mathbb{E}_2)
\end{equation*}

\subsubsection{Formule di Bayes}
\subsubsection{Prima formula}
Essendo $\mathbb{F}_1, \mathbb{F}_2,\dotsc,\mathbb{F}_n$ $n$ eventi incompatibili tali che
\begin{equation*}
\mathbb{U} = \mathbb{F}_1\cup\mathbb{F}_2\cup\dotsb\cup\mathbb{F}_n
\end{equation*}
si consideri un evento $\mathbb{E}$ tale che
\begin{equation*}
\mathbb{E} = \left(\mathbb{E}\cap\mathbb{F}_1\right)\cup\left(\mathbb{E}\cap\mathbb{F}_2\right)\cup
\dots\cup\left(\mathbb{E}\cap\mathbb{F}_n\right)
\end{equation*}
si ha
\begin{equation*}
p(\mathbb{E}) = \sum\limits_{i=1}^{n}p\left(\mathbb{E}\cap\mathbb{F}_i\right) =
\sum\limits_{i=1}^{n}\big(p\left(\mathbb{E}\mid\mathbb{F}_i\right)\cdot 
p\left(\mathbb{F}i\right)\big)
\end{equation*}

\subsubsection{Seconda formula}
Essendo $\mathbb{F}_1, \mathbb{F}_2,\dotsc,\mathbb{F}_n$ $n$ eventi incompatibili tali che
\begin{equation*}
\mathbb{U} = \mathbb{F}_1\cup\mathbb{F}_2\cup\dotsb\cup\mathbb{F}_n
\end{equation*}
sia $\mathbb{E}$ un evento tale che $p(\mathbb{E})>0$, per calcolare le probabilità condizionali si
usi
\begin{equation*}
p\left(\mathbb{F}_i\mid\mathbb{E}\right) =
\frac{p\left(\mathbb{E}\mid\mathbb{F}_i\right)\cdot p(\mathbb{F}_i)}
{\sum\big(p\left(\mathbb{E}\mid\mathbb{F}_i\right)\cdot p(\mathbb{F}_i)\big)}
\end{equation*}
